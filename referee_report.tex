
\documentclass[12pt, letterpaper]{article}
\usepackage{colortbl}

\newcommand{\blue}{\color{blue}}

\begin{document}

\begin{center}

{\bf Response to the reviewer's opinion 

on the manuscript

"Solar-Cycle Variation of quiet-Sun Magnetism and Surface Gravity Oscillation Mode" \\ by M. J. Korpi-Lagg et al. submitted to A\&A}

\end{center}

{\blue The manuscript describes a study of the f-mode properties throughout the solar cycle. It uses a novel approach of finding the regions (pixels) of minimal activity to ensure that the results are not affected by the localised spike of the magnetic field. In the second part, the authors study the f-mode signals in two particular active regions and study the amplitude of these signals in the pre-emergence stage.

The manuscript is concise, and clearly written, and the methodology is also very clear. Results are presented in a very good way so I have only very minor comments. The manuscript well meets international standards and should be accepted for publication in A&A.}

\medskip \noindent We thank the referee for the positive opinion on our manuscript, and the constructive and helpful comments provided. We have marked all changes in the manuscript with a boldface font. Below we give our detailed responses to the reviewer's comments. \medskip

{\blue The comments (all minor) in random order:}

{\blue * It is not unusual to express the frequency of oscillations in the angular frequency $\omega$, however, it is quite unusual to give also these values in $\omega$ in the units of rad/s. In the literature, the units of mHz and consequently the frequency $\nu$ are used.}

\medskip \noindent We agree, and have now used $\nu$ in mHz in the plots and text in Section 2.5, and more clearly explained the connection between $\nu$ and $\omega$ there. \medskip

{\blue * In Fig. 3 the bottom panel the background (BG) signal is not visible (it is only indicated in the legend, but I don't see the light red line anywhere). Is it because the background signal is that close to the horizontal axis? I don't really expect that, but if so, still, it should be made visible to the readers. Selection of a different colour may do.}

\medskip \noindent At this frequency, the background is, indeed, very weak, and cannot be discerned from zero in this plotting scheme. We prefer to keep the linear scale for the main illustrative purpose of Fig 3. Hence, we have removed the invisible background curve from the plot.  \medskip

{\blue * In Fig. 7, there is a sharp jump in columns from bluish to reddish in a single pixel. It probably corresponds to the transition between Q3 2018 and Q4 2018. In all other cases, the transitions as displayed on the plot are "smooth", this is the only abrupt change. Such a singularity deserves calls for explanation or interpretation.}

\medskip \noindent We did search for possible changes in the instrument configuration logs, and could only find a focus update during this specific time. We tried to correct for it, but this procedure only resulted in a similar jump on the southern hemisphere. Taken that focus updates have been occurring on other times, too, without any noticeable change in the f-mode, we concluded that the jump seen in the northern hemisphere most likely has nothing to do with this update. We have now added some text to Sect. 3.1, page 5, to discuss this. We must conclude that the cause of the jump remains enigmatic. \medskip

{\blue * On page 5, in the left column, the middle paragraph, two lines before the end, there are two "the"s following each other.}

\medskip \noindent Corrected, thank you for pointing this out. \medskip

{\blue * I somehow do not believe the claims of the authors that some of the long-term trends are due to the ageing of the instrument, because (if I understood it correctly) it should be taken care of by the "detrending" described in Section 2.3. Or is it so that only the jump in 2016 is corrected and the polynomial fits are not used in detrending?}

\medskip \noindent We correct for the jump, but do not detrend the data. Slight signs of instrument ageing are seen in the one degree magnetogram tracking data, but such instrumental effects should be evenly distributed over latitudes. This is not what is seen, however, so we do agree with the referee that instrumental effects are not the cause. We have now clarified our procedure in Sect. 2.3.

\end{document}