%AL BEGIN unnecessary stuff
\color{cyan}
\subsection*{All the stuff below this line not needed !!!}
Error definitions:\\
$\lambda$ = latitude, $t$=time, \\
$N_t$ = number of bins (time), \\
$N_\lambda$ = number of bins (latitude), \\
$N_P$ = number of \fff measurements in a patch, \\
$\overline{X}$ indicates average over time, \\
$f(\lambda,t)$ = \fff strength (unbinned)\\
$f_P(\lambda,t)$ = \fff strength in latitude / time bins (patches)\\
$\overline{f}(\lambda) = 1/N_t \sum_t f(\lambda,t)$ = temporal mean of the \fff strength (unbinned)\\ 
$f'=f-\overline{f}$ is the fluctuation around the temporal mean, i.e. the solar cycle dependence of the \fff (unbinned)\\
$\overline{f_P}(\lambda) = 1/N_t \sum_t f_P(\lambda,t)$ = temporal mean of 
\fff strength (in lat / time bins), \\
$f'_P=f_P-\overline{f_P}$ is, correspondingly, the binned fluctuation\\

%AL added this
The following 4 definitions represent what is plotted in \fig{butterfly_lat_old}:
\begin{itemize}
\item sdev(cycle) 
%AL added the subsript cyc to indicate that this sigma is computed  over the whole cycle, and to distinguish it from \sigma_P
%$\sigma(\lambda)$ 
$\sigma_{\mbox{cyc}}(\lambda)$ 
(red shaded area): this quantity displays the variation of the \fff over the solar cycle.
%AL added
the subscript 'cyc' should indicate that this variation includes also the solar cycle variation. This distinguishes it from $\sigma_P$ defined below, which does not contain the solar cycle variation.
\begin{equation}
%MJK I think subscript P is missing from here, and the need for 'cyc' I cannot understand.
%AL correct. The P was missing. I tried to explain the 'cyc'
%\sigma^2_{\mbox{cyc}}(\lambda) 
%AL commented out this:
%\sigma_P^2(\lambda)  = 
%AL and changed it to:
\sigma^2_{\mbox{cyc}}(\lambda) =
\frac{1}{N_t} \sum_{t} \left( f_P(\lambda,t) -\overline{f_P}(\lambda)\right)^2,
\end{equation}
%MJK added this
hence equalling to the rms value of $f'_P$ (without taking the square root). 
%MJK

\item min/max $\sigma_{\mbox{mm}}$: (blue shaded area): this shows the minimum and the maximum  of $f(\lambda,t) - \overline{f}(\lambda)$:
\begin{equation}
\sigma_{\mbox{mm}}(\lambda) = [ \min_t (f(\lambda,t)),  \max_t (f(\lambda,t)) ] -\overline{f}(\lambda)
\end{equation}

\item min/max (mean) $\overline{\sigma}_{\mbox{mm}}$: (green shaded area): this shows the mean minimum and maximum  of $f(\lambda,t) - \overline{f}(\lambda)$:
\begin{equation}
\overline{\sigma}_{\mbox{mm}}(\lambda) = [ \overline{ \min_t (f(\lambda,t)) }, \overline{ \max_t (f(\lambda,t)) }] -\overline{f}(\lambda)
\end{equation}

\item mean(sdev(patch)) $\overline{\sigma_P}$ (error bars): the standard deviation of the \fff strength
%AL added this
($f(\lambda,t)$)
%AL
 is computed individually for all lat / time bins. This quantity  ($=\sigma_P(\lambda,t)$) is then averaged over time:
\begin{eqnarray}
\overline{\sigma_P}(\lambda) &=&  \frac{1}{N_t} \sum_{t}  \sigma_P(\lambda,t) \mbox{, with} \\
\sigma_P^2(\lambda,t) &=& \frac{1}{N_P}\sum_{\mbox{patch}} ( f(\lambda,t)
- f_P(\lambda,t) )^2,
\end{eqnarray} 
%AL added this
where $\sum_{\mbox{patch}}$ indicates the summation over all \fff strengths within a patch, and $f_P(\lambda,t)$ being the mean \fff strength in this patch. The patch size must be chosen small enough to not contain a significant variation in latitude or solar cycle. 
This quantity $\overline{\sigma_P}(\lambda)$ is displayed as the orange shaded area in \fig{butterfly_lat}.\\
%MJK Added this
This quantity is, therefore, a mean of the $f'_{P, {\rm rms}}$ (so this time also taking the square root) values in individual bins. The problem is that the quantity we are interested in inspecting, and defining the significance of, is now defining the error bar of the measurement. 
%MJK
%AL comment:
%My definition is maybe poor. Yes, I think it should be simply called the 'average rms fluctuation of the \fff strength over 2 months' (in the above plot the patch size in time is 2 months).
\end{itemize}
\subsection*{All the stuff above this line not needed !!!}
\color{black}
%AL END unnecessary stuff

%MJK Cutting this out following the decision from yesterday
%
%AL create this plot with:
ot with:
% %run plot_fmode_butterfly.py -i ../qs_cubes --reread=False --fov 'FFOV_KX0' --show_cos=False --show_cycle=False
\colfigtwocol{butterfly_lat_FFOV_KX0}{Same as \fig{butterfly_lat}, but computed from a wedge of $\pm10^\circ$ around the $k_y$ axis.
%MJK I would not know what is a poloidal component of the f-mode.
%, representing the poloidal component.
\label{butkx0}}

\colfigtwocol{butterfly_lat_FFOV_KY0}{Same as \fig{butterfly_lat}, but computed from a wedge of $\pm10^\circ$ around the $k_x$ axis.
%, representing the toroidal component. 
\label{butky0}}

%MJK
Next we investigate whether constructing the \fff 
%power 
energy
near
the ring axes will affect the results. Such a difference could be caused by the underlying magnetic field running in a preferred direction. 
%MJK
\Fig{butkx0} shows the butterfly near the $k_x$=0 axis, $f_{k_{y}}$, tracing \fff sensitivity on the poloidal component of the
magnetic field, and \Fig{butky0} the same near the $k_y$=0 axis,
$f_{k_{x}}$, tracing the sensitivity to the toroidal field. Interestingly,
differences can be observed. First of all, the variability in
$f_{k_{x}}$ is somewhat stronger then in $f_{k_{y}}$, that
could indicate that the subsurface toroidal field is stronger
than the preferentially poloidal one. Secondly, $f_{k_{x}}$ is
better in phase with the solar cycle, although some phase shift,
but smaller, is observed. The signal is the most pronounced
at the equator. It does not increase during the 24-to-25
minimum except at higher latitudes, and no signs of quenching
are yet clearly discernible during the ascending phase of Cycle 25. 

$f_{k_{x}}$ has a clear and larger phase shift w.r.t. the solar 
cycle, especially pronounced at high latitudes. In the beginning
of the HMI data series, clear signs of quenching are still
visible at high latitudes on both hemispheres. Then enhancement is 
seen around the the solar maximum. This is followed by weak signs
of quenching during the descending and minimum phase. Enhancement
is again observed during the ascending phase of Cycle 25, especially
in the North. The Southern hemisphere shows generally weaker
patterns than the North in all 
computed
\fff 
%components.
variants.

%There are two sources of magnetic
%fluctuations: the small-scale dynamo (SSD) instability and the action of %turbulent
%convective motions on the underlying large-scale magnetic field (usually
%called tangling, REFS). These processes are inseparable and without a %clear boundary, 
%as the turbulent driving of the SSD and tangling most likely occur at %similar
%scales, akin to the convective turbulence itself being the driver of both
%effects. These effects, however, might have a different timescale, and %hence
%be distinguishable: SSD is prone to replenish the magnetic fluctuations 
%exponentially, while the tangling can be expected to be linear in time.  
%Hence, we could expect a scenario, where at the smallest scales, the magnetic
%field would be only be replenished rapidly by the SSD, while the larger
%convective cell boundaries would accumulate magnetic field from both the
% effects due to flux expulsion (REFS). Hence, one would expect no solar cycle
% dependence for the interiors of the convective cells, while a dependence
%  should be observed for the magnetic fields at scales that include also
%  the convective cell boundaries.