%\documentclass[twocolumn,linenumbers,trackchanges]{tex/aastex631}
\documentclass{aa}
\usepackage{xcolor}
\usepackage{trackchanges}
\usepackage{gensymb}
\usepackage[colorlinks,allcolors=blue]{hyperref}
\usepackage{comment}
%\input{journal} 

\usepackage{graphicx}

\definecolor{cmt}{rgb}{0.5,0.0,0.0}
\definecolor{al}{rgb}{0.6,0.2,0.0}
\definecolor{mk}{rgb}{0.4,0.4,0.0}
\definecolor{no}{rgb}{0,0.0,0.4}
\definecolor{ns}{rgb}{0.4,0.0,0.0}
\definecolor{hr}{rgb}{0.4,0.0,0.0}

\newcommand{\al}[1]{{\color{al}AL: #1}}
\newcommand{\mk}[1]{{\color{mvn}MJK: #1}}
\newcommand{\no}[1]{{\color{corr}NO: #1}}
\newcommand{\ns}[1]{\textsc{\color{al}NS: #1}} 
\newcommand{\hr}[1]{\textbf{\color{ref}HR: #1}} 
\newcommand{\todo}[1]{{\color{cmt}ToDo: #1}}
\newcommand{\cmt}[1]{
{\color{cmt}#1}
}

%\renewcommand{\fig}[1]{Fig.~\ref{#1}} 
\newcommand{\fig}[1]{Fig.~\ref{#1}} 
\newcommand{\Fig}[1]{Figure~\ref{#1}} 
\newcommand{\eq}[1]{Eq.~\ref{#1}} 
\newcommand{\figs}[1]{Figs.~\ref{#1}} 
\newcommand{\tab}[1]{Tab.~\ref{#1}} 
\newcommand{\Tab}[1]{Table~\ref{#1}} 
\newcommand{\sect}[1]{Sec.~\ref{#1}} 
\newcommand{\Sect}[1]{Section~\ref{#1}} 
\newcommand{\secs}[1]{Sections~\ref{#1}} 
\newcommand{\app}[1]{App.~\ref{#1}} 


\newcommand{\kms}{km\,s$^{-1}$}
\newcommand{\ms}{m\,s$^{-1}$}
\newcommand{\grad}{$^\circ$}
\newcommand{\carcsec}{$\mbox{.\hspace{-0.5ex}}^{\prime\prime}$}
\newcommand{\halpha}{H$\alpha$}
\newcommand{\hminus}{H$^-$}
\newcommand{\hei}{He\,\textsc{i}}
\newcommand{\sii}{Si\,\textsc{i}}
\newcommand{\fei}{Fe\,\textsc{i}}
\newcommand{\tii}{Ti\,\textsc{i}}
\newcommand{\cai}{Ca\,\textsc{i}}
\newcommand{\caii}{Ca\,\textsc{ii}}
\newcommand{\caiih}{Ca\,\textsc{ii\,h}}
\newcommand{\coi}{Co\,\textsc{i}}
\newcommand{\helixp}{\textsc{HeLIx$^+$}}
\newcommand{\imax}{{IMaX}}
\newcommand{\sufi}{\textsc{SuFI}}
\newcommand{\sunrise}{\textsc{Sunrise}}
\newcommand{\los}{${\rm LOS}$}
\newcommand{\vlos}{$v_{\rm LOS}$}

\newcommand{\IN}{internetwork}
\newcommand{\inw}{\textsl{INw}}
\newcommand{\nw}{\textsl{Nw}}
\newcommand{\brms}{$B_{\rm RMS}$}

\newcommand{\colfig}[3][1.]{\begin{figure}\centering
    \includegraphics[width=#1\linewidth,clip=TRUE]{#2}
    \caption{#3}
    \label{#2}
\end{figure}}
\newcommand{\colfigtwocol}[3][1.]{\begin{figure*}\centering
    \includegraphics[width=#1\linewidth,clip=TRUE]{#2}
    \caption{#3}
    \label{#2}
\end{figure*}}


\graphicspath{{./figures/}}

\begin{document}
%\title{Quiet-Sun Magnetism Depends on Solar Activity}
\title{Solar-Cycle Variation of quiet-Sun Magnetism and Surface Gravity Oscillation Mode}
%\shorttitle{Quiet-Sun Solar Activity}
\titlerunning{Quiet-Sun Solar Activity}

%\shortauthors{Aalto-f-mode gang.}
\authorrunning{Aalto-f-mode gang.}


\author{N. Olspert\inst{1} \and H. Raichur\inst{2} \and A. Lagg\inst{3} \and M. K\"apyla\inst{1,3,4} \and H.-L. Truong\inst{1} \and N. K. Singh\inst{3}}
\institute{Department of Computer Science, Aalto University, PO Box 15400, FI-00076 Aalto, Finland \and
IUCAA, India \and
MPS, G\"ottingen, Germany \and
Nordita, Sweden
}



\abstract{The origin of the quiet Sun magnetism is under debate. Investigating the solar cycle variation observationally in more detail can give us clues about how to resolve the controversies.}
{We investigate the solar cycle variation of the most magnetically quiet regions and
their surface gravity oscillation (f-) mode integrated power as function of latitude.}
{We use 11 years of HMI data and apply a stringent selection criteria, based on
spatial and temporal quietness, to avoid any influence of active regions (ARs).
We develop an automated high throughput pipeline to go through all available magnetogram data
and to compute the surface gravity mode power for the selected quiet regions.}
{We observe a clear solar cycle dependence of the magnetic field strength in the most
quiet regions containing several supergranular cells. For patch sizes smaller than a supergranular
cell, no significant cycle dependence is detected. The f-mode at the supergranular scale 
%MJK 
is not constant over time.
%shows a weak cycle dependence, which is not 
%completely in phase with the cycle.
%MJK Trying anew
During the late ascending phase of Cycle 24 (2011-2012), the f-mode strength is roughly constant, but starts diminishing in 2013, as the maximum of Cycle 24 is approached. This trend continues
until 2017, in the middle of which year we see hints of strengthening.
Slow strengthening continues, stronger at higher latitudes than at the equatorial regions, but the f-mode strength never returns back to the values seen in 2011-2012. Also, the strengthening trend continues past the solar minimum, to the years when Cycle 25 is already clearly ascending. Hence the f-mode behavior is not in phase with the solar cycle. 
%MJK
}
{The solar cycle dependence at the supergranular scale is indicative for the fluctuating magnetic
field being replenished by tangling from the large-scale magnetic field, and not solely due to
the action of a fluctuation dynamo process in the surface regions. 
The absence of variation at smaller scales might be an effect of the
limited spatial resolution and magnetic sensitivity of HMI.
%MJK Commenting this out now
%Similar conclusion applies
%to the variation of the surface gravity mode power, although the reason %for the phase lag
%is not yet understood. 
An anticorrelation of the f-mode strength w.r.t. the solar cycle is expected, as active regions efficiently damp it. The f-mode behavior, although showing such behavior in gross terms, is much more complex than this - in addition we see a phase shift, and different damping behavior of the ascending phases of Cycles 24 and 25. We speculate that the latter two could be related to the magnetic helicity-constrained subsurface toroidal field rather than instrumental effects.
}
%\end{abstract}

%% Keywords should appear after the \end{abstract} command. 
%% The AAS Journals now uses Unified Astronomy Thesaurus concepts:
%% https://astrothesaurus.org
%% You will be asked to selected these concepts during the submission process
%% but this old "keyword" functionality is maintained in case authors want
%% to include these concepts in their preprints.
\keywords{Sun}

\maketitle

\section{Introduction} \label{sec:intro}

%NS: adding some text which can be moved around later to a more suitable place
%NS: please modify at wish, here and elsewhere
Localized regions of intense bipolar magnetic structures, called active regions
(ARs), are seen on the solar surface. Their numbers vary periodically and trace the
butterfly diagrams which exhibit a cyclic magnetic activity of the Sun in a latitude--time
domain. Such diagrams have proved to be useful and reveal some properties of solar
magnetism, origin of which is not yet fully understood.
%Due to a number of reasons, it would be %extremely useful to identify those localized regions
%in advance where ARs are going to emerge later. 
In an observational study involving
HMI data, \cite{SRB16} reported strengthening of solar surface or the fundamental $f$-mode
about one to two days before
the formation of AR on the same corotating patch of area $(180\, \Mm)^2$,
%MJK
and quenching of the mode after the formation of the AR has been postulated
(REFS) and reported (REFS).
Very recently, another study reported similar results using a different
helioseismic technique (Fourier-Hankel method, \cite{Waidele22}).
%MJK
%Detections of such localized perturbations in the solar $f$-mode due to ARs can have
%important applications for early forecasting of the space weather. 
%These 
Detection of such localized perturbations
could also shed light
on the physics behind the formation of ARs.
%MJK
Furthermore, such behavior of the f-mode has been seen in compressible
isothermal MHD simulations (REFS). 

%MJK
The observational studies of \cite{SRB16} and \cite{Waidele22} used quiet Sun (QS)
regions on the opposite hemisphere to compare with the AR f-mode. This method
requires that a quiet patch exists on the other hemisphere, and hence 
limits the number of ARs that can be included in the hindcasting procedure. It is
also prone to be affected by the probable fluctuations in the QS
f-mode level. Although the results look promising, proper calibration with
statistically sound QS level, not just a comparison of a random QS patch on the
other hemisphere, is necessary to prove these findings robust. Also,
such a calibration procedure is necessary for increasing the sample size. 
Building such a QS calibration data product is one of the main aims of this
study: we carefully identify the quietest regions on the solar surface based on the level of magnetic activity observed in
line-of-sight (LOS) magnetograms that are readily available from HMI, and
compute the f-mode strength at the central meridian as function of latitude
and time with suitable averaging. 
%MJK

%MJK 
Building such a data product for the f-mode, requiring us to identify the most
inactive regions on the solar surface, allows us to extract statistics
of the QS magnetism as well. 
Although QS magnetism is weak, HMI provides high sensitivity, high spatial
resolution, long-term stability, and constant conditions.
It has been argued that the QS magnetic field is 
independent of the solar cycle (REFS) and some other studies have proposed
that some dependence should exist (REFS). There are two sources of magnetic
fluctuations: the small-scale dynamo (SSD) instability and the action of turbulent
convective motions on the underlying large-scale magnetic field (usually
called tangling, REFS). These processes are inseparable and without a clear boundary, 
as the turbulent driving of the SSD and tangling most likely occur at similar
scales, akin to the convective turbulence itself being the driver of both
effects. These effects, however, might have a different timescale, and hence
be distinguishable: SSD is prone to replenish the magnetic fluctuations 
exponentially, while the tangling can be expected to be linear in time.  
Hence, we could expect a scenario, where at the smallest scales, the magnetic
field would be only be replenished rapidly by the SSD, while the larger
convective cell boundaries would accumulate magnetic field from both the
 effects due to flux expulsion (REFS). Hence, one would expect no solar cycle
 dependence for the interiors of the convective cells, while a dependence
  should be observed for the magnetic fields at scales that include also
  the convective cell boundaries. We also set out to investigate these
  possible scenarios with our data products. 
%MJK

%MJK
The paper is organized as follows: in Sect.\ref{pipeline} we describe
the data, the necessary steps to prune it, and the automated pipeline we built for harvesting the data and compiling the end products, namely the QS magnetism data products, and the QS f-mode and AR data. In
Sect.\ref{results} we discuss our findings for the QS magnetism, f-mode, and the lastly present some AR data with the QS calibration
applied. 
%MJK

%Nature of AR induced $f$-mode perturbations appear complex with respect to a level
%corresponding to the quiet Sun. There is first an enhancement of
%$f$-mode power about 1-2 days before the emergence, followed by a damping of the mode
%as the AR begins to emerge; see \cite{SRB16}, for more details.
%This makes it harder to predict a newly forming ARs in close proximity to existing ARs
%which would have already caused the damping of $f$-mode in such a `crowded' environment.
%Even in the quiet phase of the Sun, local $f$-mode power displays a systematic variation
%depending on the location of the patch on the solar disk. Its power decreases as we move away
%from the disk center --- an effect which may be largely attributed to the limb-darkening.
%AL I'd rather say it's an effect of foreshortening: 
%AL: The amplitude of the oscillations in the LOS component decreases.
%\cite{SRB16} suggested the following fitting function to account for this variation
%in the $f$-mode power during the quiet phase of the Sun:
%$\zeta(\cos \alpha)=\cos \alpha [q+(1-q)\cos \alpha]$ with $q=0.5$,
%AL We updated this formula a bit after the quiet Sun calibration to account mainly for effects
%AL at high alpha. But we do not use this calibration in this paper, maybe we can skip this eq.
%where the angular distance $\alpha$ from the disk center can be expressed in terms of
%latitude $(\vartheta)$ and longitude $(\varphi)$ of point of interest as
%$\cos \alpha = \cos \vartheta \cos \varphi$.
%NS.

%NS: continued
%It would be extremely useful to understand the background evolution of quiet-Sun $f$-mode
%over solar cycles in order to more reliably predict the photospheric emergence of ARs,
%regardless of their environment, crowded or isolated. This is expected to enable us explore
%a statistically large number of ARs by studying the properties and evolution of their
%associated local $f$-mode power. This provides a good motivation to determine a butterfly
%diagram for the whole cycle, as mentioned above, but now based on the solar $f$-mode power
%from the magnetically quiet patches. Present work aims to address this by carefully identifying
%the quietest regions on the solar surface based on the level of magnetic activity %observed in
%line-of-sight (LOS) magnetograms that are readily available from HMI. 
%NS.

%MJK We would need some text about the magnetic helicity business
%MJK also to the intro.

%\begin{itemize}
%\item quiet Sun magnetism is weak. Measurement requires high sensitivity, high spatial %resolution, otherwise signal cancellation. Variations in quiet Sun magnetism are even %tougher to detect. Require long-term stability, constant conditions, HMI offers this, %now 1 solar cycle in orbit. 
%\item what is the quiet Sun? simple: No AR. 98\% of the solar surface are in this %state. But even the quiet regions are structured: network / internetwork. 
%\item what is expected to vary with solar cycle? network, IN, both, or nothing? A few %words about the expectations.
%\item argue why taking into account network is important: the IN flux is advected %towards the network boundaries. A cycle variation of the IN flux therefore should be %reflected in a variation of the network flux. Network accumulates IN flux and %therefore acts as a memory of what happened in the IN. Helps to overcome detection of %the weak IN fields (difficult enough), even more difficult to study variations of %these weak fields.
%\item HMI: compared to Hinode / ground based / Sunrise: not the most sensitive %instrument to B. But: Stability allows averaging. 8 hr time and spatial averaging %allows to boost S/N ratio (can we give a number here?)%
%
%\end{itemize}
%
%\cite[]{2019LRSP...16....1B}
%Luis Talk PHI meeting: 38\% of the flux emerging in the IN makes it to the network. %Network flux is constant, the IN flux to the network accumulates quickly to network %flux.
%
%\cite[]{2015ApJ...806..174J}
%
%\cite[]{2013A&A...555A..33B}
%
%\cite[]{2021arXiv210508657F}
%
%\cite[]{2021arXiv210514533R}
%
%\cite[]{ballot2021changes}


\section{Observations}\label{pipeline}

Our analysis is based on data from the Helioseismic Magnetograph and Imager \cite[HMI,][]{2012SoPh..275..207S,2012SoPh..275..229S} on board the Solar Dynamics Observatory \cite[SDO,][]{2012SoPh..275....3P}. We use two standard data products: (i) full-disk line-of-sight (LOS) magnetograms, computed every 720\,s by comining filtergrams obtained over a time interval of 1260\,s (\texttt{hmi.M\_720s}), and (ii) full-disk LOS dopplergrams, computed every 45\,s from six positions across the nominal 6173.3\,\AA{} spectral line (\texttt{hmi.V\_45s}). We processed the two data sets in a semi-automatic pipeline (see \fig{pipeline}), optimized for obtaining reliable information abount the magnetic field in the quiet Sun regions and for a robust computation of the $f$-mode power from the dopplergrams. The left tree in \fig{pipeline} describes the pipeline used for the magnetograms, the right tree for the dopplergrams.

\colfig{pipeline}{Data pipeline}

\subsection{Magnetograms}

The first data product, full-disk line-of-sight (LOS) magnetograms, provides a direct measurement of the variability of the quiet-sun magnetism during a solar cycle. To enhance the signal-to-noise ratio in the LOS magnetograms we performed a newly developed algorithm for spatial and temporal averaging: The full-disk HMI magnetograms starting from 27-Apr-2010 and ending at \todo{01-Oct-2021} were downloaded from the Joint Science Operation Center (JSOC) hosted at Stanford University (http://jsoc.stanford.edu) to a temporary storage (see \fig{pipeline}, 'mahti storage') and tracked at full spatial resolution for 8 hours to compensate for the solar rotation ('full disk tracking').
%AL @Nigul: Can you explain the differential rotation model used?
%NO As much as I remember we omitted taking into account diff. rot. as it was marginal during 8hrs
The step between tracked sequences was 4 hours, so that a total of 6 tracked sequences were gathered per one day, resulting in more than $24\,000$ tracked sequences.

From each tracked sequence, we extracted two data products by dividing the visible solar disk  between latitudes and longitudes from $-80\degree$ to +$80\degree$ into (i) $64\times 64$  overlapping patches of $15\degree$ (in solar latitude and longitude)  and (ii) $180\times 180$ patches of $1\degree$.
Every of these patches therefore contains a space-time cube of LOS magnetograms at full spatial resolution at a 12-minute cadence, allowing to compute the statistical properties (mean, standard deviation, skewness, kurtosis) of the following parameters: $<B>$ -- mean of the magnetic field strength, $<|B|>$ -- mean of the absolute value of the magnetic field strength, and $\sqrt{<B^2>}$ -- root mean square of the magnetic field strength (\brms{}). The $15\degree$ patches (i) are large enough to cover several supergranulation cells containing network and \IN{} fields \cite[]{2010LRSP....7....2R} with a typical size of 30--35\,Mm (we refer to them as \nw{} cubes), and the $1\degree$ patches (ii) are small enough that some of them lie completely in the \IN{} (\inw{} cubes). 
The statistics for each patch were stored as data products (see also \fig{pipeline}) including the information about latitudinal and longitudinal position as well as the Carrington longitude for network and \IN{}. We refer to these maps as the \nw{} and the \inw{} statistical maps.

 
% \begin{itemize}
% 	\item $\mu_B$ -- Mean of the magnetic field strength
% 	\item $\sigma_B$ -- Standard deviation of the magnetic field strength
% 	\item ${\rm skew}_B$ -- Skewness of the magnetic field strength
% 	\item ${\rm kurt}_B$ -- Kurtosis of the magnetic field strength
% 	\item $\mu_{|B|}$ -- Mean of the absolute value of the magnetic field strength
%     \item $\sigma_{|B|}$ --Standard deviation of the absolute value of the magnetic field strength
%     \item ${\rm skew}_{|B|}$ -- Skewness of the absolute value of the magnetic field strength
%     \item ${\rm kurt}_{|B|}$ --Kurtosis of the absolute value of the magnetic field strength
% 	\item \brms{} -- Root mean square of the magnetic field strength
%     \item $\sqrt{\sigma:{B^2}}$ -- Root mean square of the standard deviation of the magnetic field strength
%     \item $\sqrt{{\rm skew}_{B^2}}$ -- Root mean square of the skewness of the magnetic field strength
%     \item $\sqrt{{\rm kurt}_{B^2}}$ -- Root mean square of the kurtosis of the magnetic field strength
% \end{itemize}

\subsection{Quiet region selection\label{quietregion}}

From the parameters computed from the cubes the root mean square of the magnetic field strength (\brms{}) turned out to be the best tracer for determining the magnetic activity level. It could clearly distinguish between $15\degree$ patches containing active regions, plage, enhanced network and quiet network. Also, it depicted very well the low-field \IN{} regions.

The analysis of the solar-cycle variation of the quiet-sun magnetism required a careful selection of the most quiet regions, defined as being free of enhanced solar activity. We therefore searched for the minimum value of \brms{} in both, the \nw{} and the \inw{} statistical maps, on a latitudinal grid with a $10\degree$ spacing fulfilling the following additional criteria:
\begin{enumerate}[(i)]
	\item\label{c1} the most quiet pixel must be within $\pm10\degree$ around the central meridian,
	\item\label{c2}  this pixel must belong to the 50\% most quiet pixels of the month,
	\item\label{c3}  this pixel must be the most quiet pixel within a 4-day interval.
\end{enumerate}

Criterion (\ref{c1}) was chosen to get the strongest magnetic field signal along the central meridian. (\ref{c2}) guarantees an equal distribution of quiet pixels over the 11-year period of available HMI measurements, and (\ref{c3}) ensures that the quiet pixels for the 1-month period do not originate from the same supergranular structure, since the dynamical evolution time of the supergranulation lies between 24 and 48\,h \cite[]{2010LRSP....7....2R}. The result of this selection were two time series of the \brms{} for the most quiet patches in the network and the \IN{} regions. Note that the this selection also efficiently removes the 24\,h modulation present in the HMI magnetograms.

\subsection{Correction for HMI sensitivity change\label{sensicorr}}

The 11-year temporal evolution of \inw{} time series{the \brms{}} value revealed clearly a change in the HMI observing mode, performed on 13-Apr-2016. On this day, HMI switched to a more efficient observing mode \cite[see][]{2018SoPh..293...45H,2014SoPh..289.3483H,2016SoPh..291.1887C}. By combining both HMI cameras to determine the vector-field observables the cadence for full-disk magnetograms could be reduced from 135\,s (observational mode MOD-C) to 90\,s (MOD-L). This reduced the noise level for Stokes $V$ measurements by 17\%, resulting in a decrease of the noise level in the \los{} magnetograms by 5\%.

For the long-term study presented in this paper, we need to correct for this sensitivity change. A very accurate correction method can be derived from the \inw{} time series: since it contains only the most quiet pixels over a certain latitude region and time, the sensitivity change results in a step function. The value of the step was determined by fitting a polynomial to the \brms{} values determined from the \inw{} time series plus a Heaviside step function, centered at the date of the mode change. We used the Bayesian information criterion \cite[BIC,][]{Stoica2004} to determine the degree of the polynomial, which lies between 1 and 6 for the various latitudes. We want to note that the retrieved amplitude of the Heaviside step function is only weakly dependent on the degree of the polynomial. This fitting is exemplified in \fig{fit-offset} for the solar latitude 0$^\circ$, where the minimum value for the BIC was reached for a fit with a polynomial of degree 4. The so determined amplitude of the Heaviside step function is added to the data points after 13-Apr-2016 for all data presented in this paper.



%MJK Remove the title; typeset y-axis in latex format; insert x-axis label
%MJK RMS we usually write with lower case, hence unify RMS -> rms
%MJK throughout
\colfig{fit-offset}{Determination of the correction for the HMI sensitivity change: the observing mode change on April 13 2016 causes a discontinuity in the level of \brms{} values of the \IN{} data. The original data are displayed in with the dark red and black dots, the corrected data with the light red dots. The dashed line indicates the polynomial fit of degree 4 used to obtain the offset.}




% \cmt{
% Key issue: 
% How to obtain the most reliable HMI data product telling us about the variability of the quiet Sun magnetism during a solar cycle? Idea: combine spatial and temporal averaging, and analyze the statistics in space-time cubes to determine the level of quietness as a function of latitude. This requires the magnetic field information coming from specro-polarimetry. 

% But first the boring stuff: How to obtain a clean, 11-year long data set:

% About the long term trend: The jump in the data happened on 13 April 2016 (search for this date in \cite{2018SoPh..293...45H}). There it states:
% \begin{itemize}
% \item Standard HMI observations were initially obtained with a framelist called Mod-C that
% repeated every 135 seconds. Mod-L, a 90-second FTS, replaced Mod-C on 13 April 2016.
% The two versions of Mod-C have FTS ID 1001 or 1021; the Mod-L HFTSACID is 1022.
% Some calibration framelists changed when the standard sequences changed.
% \item Since 13 April
% 2016, filtergrams from the two cameras have been combined to compute the vector magnetic
% field \cite[]{2014SoPh..289.3483H,2016SoPh..291.1887C}
% \item On 13 April 2016, after the prime mission ended, HMI switched to
% a faster sequence, FTS ID 1022, also known as Mod-L. The Mod-L sequence requires that
% images from both cameras be combined to determine the vector-field observables. 
% \end{itemize}
% Mod-L description here: http://hmi.stanford.edu/hminuggets/?p=1596: \textsl{
% Mod-L provides all of the filtergrams necessary to compute the Stokes parameters [I, Q, U, V] in 90 seconds, instead of the 135 seconds required for Mod-C. Thus Mod-L increases the maximum temporal resolution for measuring full Stokes parameters. At the same time, it decreases the noise because twice as many filtergrams are available. The 45s data products from the front camera are also improved; since there is no longer a 135s period in the instrument configuration, the corresponding peak in the Doppler power spectrum is now gone. The new 90s period is at the Nyquist frequency of the 45s-cadence Doppler data. The Stokes measurement is normally averaged over time (nominally 720 seconds) to derive [I, Q, U, V], and we now combine three times as many CP and 1.5 times as many LP measurements. Table 1 compares the Mod-C and Mod-L observations.}
% }



\subsection{Dopplergrams}

The second data product used in this paper are the \los{} velocity maps. The goal is to compute the power of the surface gravity mode, the so-called $f$-mode power, as an independent measure to quantify the solar cycle variation.

The dopplergram data is hosted in the German Data Center for SDO (GDC-SDO) on a server at the Max Planck Institute for Solar System Research (MPS Göttingen, Germany) whereas the analysis is executed in the CSC supercomputing environment (Aalto, Finland). We have utilized  a function-as-a-service client based on funcX \cite[]{chard20funcx} for accessing the required data in the database server. We have developed functions based on funcX API and deployed the functions in the MPS environment. These functions leverage the mtrack\footnote{\url{http://hmi.stanford.edu/teams/rings/modules/mtrack/v25.html}} command to prepare the dopplergram cubes within the GDC-SDO environment and subsequently transfer them to the CSC environment. The coordinates of the selected quiet-sun regions were sent to the funcX service, which invokes suitable functions to automate the data retrieval and movement.
%{the using remote function call service funcX \cite{chard20funcx} to MPS server where the dopplergram database is located. Next, using mtrack command the tracked dopplergram cubes were retrieved and sent back to CSC supercomputing environment.}

%AL: TODO details of the tracking: Central coordinate from quiet region selection, Harsha's parameters. $300\times 300$ pixels, 8 hrs ... 

\subsection{Computation of the $f$-mode power}

In the CSC environment, subsequent processing involved calculating the 3d power spectra for each dopplergram cube and averaging the spectra azimuthally in $k_x, k_y$ plane for each constant $\omega$ (see \fig{ring_diagram}). Such averaging significantly reduced the noise level leading to smooth one dimensional $k-\omega$ spectra. Such averaging is justified for the quiet-sun spectra, as the ring diagrams are radially symmetrical w.r.t. $\omega$ axis. From the obtained spectra $f$-mode was easily separable from the rest of the modes and integration done over $k$ and $\omega$, to obtain full power of the $f$-mode. 
Integration range in $k$ space was selected for each constant $\omega$ in a following way: first the maximum of $f$-mode $k_{\rm max}$ and the minimum between $f$-mode and first $p$-mode $k_{\rm start}$ were detected; $k_{\rm start}$ was chosen as the start of integration range and the end of the integration range $k_{\rm end}$ was set as $k_{\rm end}=k_{\rm max}+2(k_{\rm max}-k_{\rm start})$; $k_{\rm end}$ chosen in such a way guarantees that the integration range is sufficiently wide to cover significant part of the $f$-mode power in $k$-space.
Integration range in $\omega$ space was chosen between values starting from 14.45 and ending at 29.03, where the significant part of the $f$-mode power resides. After completing described steps we had collected total of \todo{18327 number to be updated for final version!} $f$-mode powers for patches close to central meridian over all latitudes covering the full solar cycle.

\begin{figure}\centering
	\includegraphics[width=1.0\linewidth,trim={0cm 0.4cm 0cm 0.3cm},clip=TRUE]{ring_diagram}\\
	\includegraphics[width=1.0\linewidth,trim={0cm 0cm 0cm 1cm},clip=TRUE]{az_avg_spec}
	\caption{Top: example of quiet-sun ring diagram at $\nu=3.56$. Bottom: spectrum obtained by azimuthally averaging the ring diagram. The red vertical line marks the position of the maximum of the $f$-mode and the black vertical lines the range of integration.}
	\label{ring_diagram}
\end{figure}

\subsubsection*{Orbital correction of the $f$-mode power}

%AL We have to be consistent. Do we call it f-mode power, strength or area?
%NS: strength is perhaps better; we could mention somewhere the strength and power
%NS: are being used interchangeably, if needed
Since the $f$-mode power is computed from the \los{} dopplergrams, its value depends strongly on the viewing geometry. This dependence,  roughly following the cosine of the solar latitude for data taken at the central meridian, is additionally modulated by the orbital motion of the Earth around the Sun, which changes the viewing angle at any given solar latitude by $\approx\pm7^\circ$ during one year. We compensated for this periodic variation by fitting the parameters ($x_0(\lambda), ..., x_3(\lambda)$) of the following function to all observations of a given latitude $\lambda$:
\begin{equation}
\label{eq:orbitcorr}
A_{\mbox{corr}}(\lambda) = x_0(\lambda) (  2 ( (1+\cos(\alpha+x_1(\lambda)))/2)^{x_3(\lambda)}-1   )+ x_2(\lambda),
\end{equation}
with $\alpha$ being the phase angle of the Earth defined as the cotangens computed from the $x,y$ barycentric position of the Earth.
This correction $A_{\mbox{corr}}(\lambda)$ is then subtracted from the computed $f$-mode power, which efficiently removes any yearly variation. 



%AL Mention also the focus change in 2018-10-16, leading to a change in the \los{} maps.

\section{Results}\label{results}

After applying the selection criteria described in \sect{quietregion} and the subsequent correction for the HMIS sensitivity change (\sect{sensicorr}) we obtain the dependency of the \brms{} values of the most quiet region at the central longitude as a function of heliographic latitude and time. \fig{Brms-lat00-percentile} presents the \brms{} values from May 2010 until October 2021 for the heliographic latitude $\lambda = 0\degree$. The individual data points (red dots) are computed over an area of 15$^\circ$ in latitude and longitude and over a time of 8 hours. At disk center, this corresponds to an area of $\approx 180 \times 180$\,Mm$^2$, and therefore contains 30--40 supergranular cells. The \brms{} value therfore contans network and \IN{} fields.
Typical magnetograms at disk center at solar minimum (July 2010) and maximum (Nov 2014) are presented in \fig{2010-07-21_2014-11-09}. There is no obvious difference between the two maps: both show similar minimum and maximum field strengths, and the visible impression of the supergranular cells with the strong field patches of both polarities surrounding the \IN{} regions is indistinguishable.

\fig{Brms-lat00-percentile} clearly indicates the solar cycle dependence of the quiet-sun magnetic fields, consisting of network and \IN{}. The \brms{} values peak in the second half of 2014, close to the declared maximum od solar cycle \#24 (April 2014, \todo{REF!}. The variation is statistically significant, as indicated by the the mode and the 97\% percentiles, computed by fitting a log-normal distribution over a 1-year sliding window.

A similar plot is presented in \fig{Brms-lat00-percentile-IN}, but now the \brms{} value was computed only for a $1\degree$ window in latitude and longitude, corresponding to an area of only $12\times 12$\,Mm$^2$ at disk center. The quiet-region selection criteria (\sect{quietregion}) guarantees, that this patch lies fully within the \IN{}, and is not `contaminated' by network fields. Clearly, there is no dependence of \brms{} with solar cycle detectable. The slight increase over the 11-year period might be an effect of the aging of the HMI instrument and the resulting decrease in the sensitivity for magnetic field measurements.

The average \brms{} value in the \IN{} is in $\approx6$\,G. Using the area of one HMI pixel this corresponds to a magnetic flux of $3\times 10^16$\,Mx. 

\colfigtwocol{2010-07-21_2014-11-09}{Comparison of two disk-center magnetograms at solar minimum (2010-07-21) and solar maximum (2014-11-09). The maps show the first frame of the 8\,h data cube, the animation is in the online material.}

\colfig{Brms-lat00-percentile}{\brms{} determined from the most quiet patches from 2010 to 2021 at disk center. The patches of 15$^\circ$ size in longitude and latitude contain network and \IN{} fields. The solid red lines display the 97\% percentile level, the dashed, gray line the mode of a log-normal fit to yearly-binned data moved in a sliding window of 100 days length.}

\colfig{Brms-lat00-percentile-IN}{Same as \fig{Brms-lat00-percentile}, but for \brms{} determined from patch size of only of 1$^\circ$ in longitude and latitude, corresponding to \IN{} patches. }


%\subsection{Butterfly}
\subsection{Quiet Sun f-mode butterfly diagram}\label{qsf}



Definitions:\\
$\lambda$ = latitude, $t$=time, \\
$N_t$ = number of bins in time \\
$N_\lambda$ = number of bins in latitude, \\
$n_t$ = total number of time points at a certain latitude, \\
$N_P$ = number of $f$-mode measurements in a patch, \\
$\overline{X}$ indicates average over full time span, \\
$\left< X \right>$ indicates an average in a selected bin, \\
%AL: The <> would remove the necessity for the subscript _P, 
% which meant exactly the same
% I removed all the _P and replaced it by <>
$f(\lambda,t)$ = $f$-mode strength \\
%AL removed _P
%$\left<f_P\right>=1/N_P \sum_{N_P} f (\lambda,t)$ = mean $f$-mode strength 
$\left< f \right>=1/N_P \sum_{N_P} f (\lambda,t)$ = mean $f$-mode strength calculated in patches of $N_t$ and $N_\lambda$ \\
$\overline{f}(\lambda) = 1/n_t \sum_t f(\lambda,t)$ = temporal mean of the $f$-mode strength, \\
%MJK This is what is plotted in the butterfly, and should appear in title/caption.
%MJK It is further binned for plotting, the explanation of which is still missing, but this should not be here, but in the figure caption.
$f'=f-\overline{f}$ is the fluctuation around the temporal mean, i.e. the solar cycle dependence of the $f$-mode.\\
%AL: In the butterfly we plot the binned f'. It is computed like this:
$\left< f' \right> = \left< f - \overline{f} \right> $ is the same as $f'$, but computed in $N_\lambda \times N_P$ bins. This is the quantity we show in the left panel of \fig{butterfly_lat}.
%MJK Added
We assume that the solar cycle variation is a sinusoidal function, the amplitude of which can be approximated as 
%AL Would it be more correct if we also use <> around the quantity f'_A?
$f'_{\rm A} (\lambda) = 
\sqrt{2} \sqrt{\left<{f'^2}\right>}$. 
The average over the bin is necessary to remove the noise from the estimate.
%AL added
This quantity $f'_{\rm A} (\lambda)$ is plotted as the orange shaded area in the right panel of \fig{butterfly_lat}.

One measure of the noise contained in the $f$-mode strength is the standard deviation of the fluctuations
around the signal. This is calculated in each bin as
\begin{equation}
\sigma= \sqrt{\frac{1}{N_P}\sum_{\mbox{patch}} ( f(\lambda,t)
- \left<f_P\right>)^2}.
\end{equation}
From this we form the average error at each latitude by taking an average over time and denote as $\overline{\sigma}$. 
%AL added
This quantity is plotted as the error bar in the right panel of \fig{butterfly_lat}.

%NO Adding some words how we verified significance
As the average error described above does not necessarily prove that the patterns seen in the left panel of \fig{butterfly_lat} are statistically significant, we did more detailed analysis and estimated the significance of $\left< f' \right>$ for each patch separately. The following test was made: For all quiet-sun patches we computed the power of noise in the azimuthally averaged spectra similarly as for $f$-modes. We assumed that noise level is independent of $k$ and selected the region towards higher $k$ values, where the $f$-mode is already decayed. Then we calculated fluctuations in the noise power $\left< f'_{\rm noise} \right>$ in an identical manner as for $\left< f' \right>$.
The analysis showed that for all patches $\left< f'_{\rm noise} \right>$ was an order of magnitude weaker than the corresponding $\left< f' \right>$.



\colfigtwocol{butterfly_lat}{Butterfly diagram of $f$-mode strength computed from azimuthal average of the ring diagram (left panel). The latitudinal average over time, displayed in the right panel, was removed for better visibility of the variation. The error bar, plotted on the latitudinal average of the $f$-mode strength, represents the time averaged standard deviation of the fluctuations of the $f$-mode strength around the signal for every bin ($\overline{\sigma (\lambda)}$). The orange shaded area shows the amplitude of the solar cycle variation ($f'_A(\lambda)$). }
%The horizontal lines in the colorbar indicate $\pm \overline{\sigma_P}(\lambda)$.

\colfigtwocol{butterfly_lat_FFOV_KX0}{Same as \fig{butterfly_lat}, computed from a wedge of $\pm10^\circ$ around the $k_y$ axis, representing the poloidal component.\label{butkx0}}

\colfigtwocol{butterfly_lat_FFOV_KY0}{Same as \fig{butterfly_lat}, computed from a wedge of $\pm10^\circ$ around the $k_x$ axis, representing the toroidal component. \label{butky0}}

%NS: added following for butterfly-figure
\Fig{butterfly_lat} displays the butterfly diagram of the 
%MJK
%quiet-Sun 
QS
$f$-mode power determined
from $15\degree \times 15\degree$ patches that showed only weak, small-scale magnetic structures.
%MJK
To compute the f-mode power, we have used the whole ring diagram data.
%MJK
Temporally averaged profile in latitude is removed for highlighting the variation in the
$f$-mode power. The period covered is somewhat larger than the 11 year solar cycle~24, which
began to show some magnetic activity from early 2010; this cycle ended by around 2020.
Broadly, we find an anti-correlation of the $f$-mode strength with the activity cycle of the
Sun --- during the solar minimum and early rising phase ($\sim 2010-12$), the $f$-mode strengths
are larger, and around cycle maximum\footnote{the Sun reached its peak magnetic activity
in terms of number of sunspots during the cycle~24 in April 2014.} to the declining phase,
the $f$-mode power is suppressed.
%AL 
%This correlation is strongest at the solar equator, and decreases with higher latitudes.
%NS: modified a bit so that it is easier to relate with magnetic field which are
%NS: away but near the equator, please check
This correlation is strongest near the solar equator, within about $\pm 20\degree$ in
latitude, and it decreases with higher latitudes.
%AL
Interestingly, we observe a temporal shift in the $f$-mode butterfly diagram with respect to the
solar cycle, such that the minimum of the $f$-mode occurs later, i.e., in the declining phase of
%MJK
%the cycle.
Cycle 24, roughly around the end of the year 2016.
%NS.


%MJK New para; describing the rest of our observations.
During mid-2017, we see weak strengthening of the f-mode strength to commence. During the minimum (2018-2019) between the Cycles 24 and 25, the strengthening appears stronger in the higher latitudes. It also continues after the minimum, when Cycle 25 has already started its ascending phase. What is also noteworthy is that the f-mode strength never rises as high as the values observed during the ascending phase of Cycle 24 (2011-2012).
%MJK

%MJK We also need to show the scatterplot of Brms vs. f-mode per year.

%MJK
Next we investigate whether constructing the f-mode power near
the ring axes will affect the results. Such a difference could be caused by the underlying magnetic field running in a preferred direction. 
%MJK
\Fig{butkx0} shows the butterfly near the $k_x$=0 axis, $f_{k_{y}}$, tracing f-mode sensitivity on the poloidal component of the
magnetic field, and \Fig{butky0} the same near the $k_y$=0 axis,
$f_{k_{x}}$, tracing the sensitivity to the toroidal field. Interestingly,
differences can be observed. First of all, the variability in
$f_{k_{x}}$ is somewhat stronger then in $f_{k_{y}}$, that
could indicate that the subsurface toroidal field is stronger
than the preferentially poloidal one. Secondly, $f_{k_{x}}$ is
better in phase with the solar cycle, although some phase shift,
but smaller, is observed. The signal is the most pronounced
at the equator. It does not increase during the 24-to-25
minimum except at higher latitudes, and no signs of quenching
are yet clearly discernible during the ascending phase of Cycle 25. 

$f_{k_{x}}$ has a clear and larger phase shift w.r.t. the solar 
cycle, especially pronounced at high latitudes. In the beginning
of the HMI data series, clear signs of quenching are still
visible at high latitudes on both hemispheres. Then enhancement is 
seen around the the solar maximum. This is followed by weak signs
of quenching during the descending and minimum phase. Enhancement
is again observed during the ascending phase of Cycle 25, especially
in the North. The Southern hemisphere shows generally weaker
patterns than the North in all f-mode components.

%AL removed this figure
%\colfig{butterfly_lat_old}{Old version of the diagram, showing the attempts to define the error, and matching the color description from equations (1)-(4).}

\subsection{AR f-mode with QS calibration}\label{arf}

%MJK Added
In this section we compare the f-mode time evolution in 
two ARs (11130 in \Fig{AR11130_track} and AR11105 in \Fig{AR11105_track}), that were reported to show an enhancement both in \cite{SRB16} and \cite{Waidele22} w.r.t. a QS control patch
in the opposite hemisphere. The main difference in our analysis
is that the f-mode strength is now normalized to the QS level
at the specific averaged time bin and the averaged latitudinal
value of the tracked cubes. 
f-mode strength of 1 is hence
equivalent to that QS level, and values stronger/weaker 
indicate enhancement/quenching of the f-mode w.r.t. QS.
Another difference is that we use the
full ring diagram when computing the f-mode strength, while the
other papers used only $k_x=0$ cuts. We have, however, performed
analysis with $k_x=0$ and $k_y=0$ cuts, and see no significant
difference between the two, unlike in the QS f-mode. 
%MJK Should somewhere discuss why this is.

As per the longitudinal normalization, we use two different
variants. The blue lines and dots indicate data that are
calibrated with the exact same cosine function than in the
aforementioned papers, and the orange lines and dots show
data with our Gaussian fit to the QS data. 
%MJK Needs to be added
The gray shaded areas indicate the QS (some) $\sigma$ level
QS variability, and hence give an indication of the significance
of the AR signal w.r.t the QS one.
The difference between these two calibrations is evident from the
rightmost panels of the f-mode butterflies. The cosine calibration
curve (green) does not fit the data well at larger longitudes,
but is decreasing there less strongly than the data. 
The Gaussian fit to the QS
data represents it better, with the consequence that the
employed calibration curve (black) decreases steeper toward the
limb. 

As can be seen
from the AR f-mode plots, the effect of the Gaussian calibration is
to flatten the region of postulated enhancement, in the case of AR11130
removing any signs of it. This analysis shows that the transient
and weak enhancement signal is very sensitive to the calibration
method used. Moreover, the enhancement level, only visible
when the cosine calibration is used, is of the same order 
in magnitude than
the QS variation, which poses an even more severe difficulty in
detecting the enhancement.
%MJK

%AL adding two AR plots as placeholders. Needs to be discussed what should be shown
\colfig{AR11130_track}{$f$-mode strength (top) and $B_{\mbox{rms}}$ (bottom) for AR11130. Orange: quiet-Sun calibration, blue: cosine calibration.}
\colfig{AR11105_track}{Same as \fig{AR11130_track}, but for AR11105.}
%AL

\section{Discussion}

%NS: added for anticorrelation; could be speculative, so please check and change as needed
%NS: a bit rushed writing as well, but it can be polished later
Anticorrelation of the $f$-mode strength with the solar cycle as seen in \fig{butterfly_lat}
is expected, as the magnetic fields of ARs tend to absorb the modes resulting in the
$f$-mode damping which has been reported in a number of earlier works
\citep{Cally+94,CB97,SRB16}. Transient effects, such as the strengthening of the mode
prior to emergence of ARs as found in \citet{SRB16}, are washed out in such a study aimed
at global behavior of $f$-mode over the whole cycle. Relevant effect here is the damping
caused by the presence of ARs. This is a nonlocal effect where quiet patches in the
vicinity of ARs too display the damping of the $f$-mode. This hypothesis was tested
by a detailed analysis of a magnetically quiet patch next to AR~12529; figures 11 and 12
in \citet{SRB16}. This is further established in the present work, where, though only the
quiet patches are selected in the analysis, damping around the time of solar maximum is
nevertheless observed, as there are a large number of ARs present on the surface during this
phase, thus nonlocally affecting the strengths of the $f$-mode.
%NS.

%NS: adding for now as a subsection, but can be changed later
%NS: hasty writing. will revise later, but essential point i thought is made
%NS: see if you agree, feel free to remove if you don't
%MJK I do agree with the importance of the toroidal subsurface 
%MJK (horizontal) field. In the patches that are used to compute
%MJK the f-mode from, there are no AR vertical fields, but 
%MJK magnetic fluctuations from SSD and tangling from the large-scale
%MJK horizontal field. If the latter is stronger, then one might
%MJK expect what you describe as the delayed damping due to the
%MJK late-rejuvenated large-scale field due to the helicity 
%MJK conservation. This would lead to the interesting conclusion
%MJK that perhaps the subsurface toroidal field is somehow much
%MJK stronger during Cycle 25 than Cycle 24. But, instrumental
%MJK degradation could easily produce something similar, hence
%MJK I think we cannot make too strong statements.
%AL The new KXY0 butterflies show exactly this: The toroidal field seems to be
%AL stronger in cycle 25 (reduced f-mode around 2021 compared to 2010).

\subsection{Dynamo origin of the shift in the $f$-mode butterfly diagram}
We speculate here on the origin of shift observed in the $f$-mode butterfly diagram
with respect to the solar cycle; see \fig{butterfly_lat}.
Series of earlier numerical and observational works have established that the $f$-mode can be
significantly perturbed in the presence of magnetic fields \citep{S+14,S+15,SRB16,S+20}.
%NS: appears too much of self-citation above; will select only most relevant ones later
In turbulent dynamo theory, the magnetic fields grow under the constraint of magnetic helicity
conservation, which leads to a bihelical spectrum of magnetic helicity. Accumulation of small-scale
magnetic helicity leads to the quenching of large-scale dynamo. In order to further grow its large-scale
magnetic field, the system must shed its small-scale helicity. Being an open system, the Sun
may have fluxes of magnetic helicity where ARs could play a vital role in removing the magnetic
helicity from small-scales, thus leading to a rejuvenation of large-scale dynamo.

It is thus expected in this scenario that the diffuse large-scale component of the magnetic field
becomes stronger \emph{after} the maximum of the magnetic cycle, as the cycle-maximum corresponds
only to the largest number of ARs during that cycle, and not necessarily to the strongest field
strengths. Such diffuse fields will suppress the vertical motions as has been seen in \citet{S+15}
based on idealized simulations, and will therefore further damp the $f$-mode.
To argue more simply, let us consider ARs as sites of vertical fields which maximize during the
solar maximum. On the other hand, diffused fields, that are rejuvenated after the cycle-maximum due
to AR-driven shedding of small-scale magnetic helicity, may be modelled in terms of horizontal
magnetic fields. Whereas both vertical as well horizontal fields lead to the damping of the
$f$-mode, the mechanisms involved are different: AR or the vertical fields couple different layers
of the Sun and act as absorbers of the surface mode power, the horizontal fields simply suppress the
vertical motions leading to lesser amplitudes for the $f$-mode.
%NS.

\begin{acknowledgements}
All SDO data used are publicly available from the Joint Science Operations Center (JSOC) at Stanford University supported by NASA Contract NAS5- 02139 (HMI), see http://jsoc.stanford.edu/. 
\todo{German DataCenter?}
\end{acknowledgements}

\bibliography{main}{}
\bibliographystyle{aasjournal}

\end{document}

%AL BEGIN unnecessary stuff
\color{cyan}
\subsection*{All the stuff below this line not needed !!!}
Error definitions:\\
$\lambda$ = latitude, $t$=time, \\
$N_t$ = number of bins (time), \\
$N_\lambda$ = number of bins (latitude), \\
$N_P$ = number of $f$-mode measurements in a patch, \\
$\overline{X}$ indicates average over time, \\
$f(\lambda,t)$ = $f$-mode strength (unbinned)\\
$f_P(\lambda,t)$ = $f$-mode strength in latitude / time bins (patches)\\
$\overline{f}(\lambda) = 1/N_t \sum_t f(\lambda,t)$ = temporal mean of the $f$-mode strength (unbinned)\\ 
$f'=f-\overline{f}$ is the fluctuation around the temporal mean, i.e. the solar cycle dependence of the $f$-mode (unbinned)\\
$\overline{f_P}(\lambda) = 1/N_t \sum_t f_P(\lambda,t)$ = temporal mean of 
$f$-mode strength (in lat / time bins), \\
$f'_P=f_P-\overline{f_P}$ is, correspondingly, the binned fluctuation\\

%AL added this
The following 4 definitions represent what is plotted in \fig{butterfly_lat_old}:
\begin{itemize}
\item sdev(cycle) 
%AL added the subsript cyc to indicate that this sigma is computed  over the whole cycle, and to distinguish it from \sigma_P
%$\sigma(\lambda)$ 
$\sigma_{\mbox{cyc}}(\lambda)$ 
(red shaded area): this quantity displays the variation of the $f$-mode over the solar cycle.
%AL added
the subscript 'cyc' should indicate that this variation includes also the solar cycle variation. This distinguishes it from $\sigma_P$ defined below, which does not contain the solar cycle variation.
\begin{equation}
%MJK I think subscript P is missing from here, and the need for 'cyc' I cannot understand.
%AL correct. The P was missing. I tried to explain the 'cyc'
%\sigma^2_{\mbox{cyc}}(\lambda) 
%AL commented out this:
%\sigma_P^2(\lambda)  = 
%AL and changed it to:
\sigma^2_{\mbox{cyc}}(\lambda) =
\frac{1}{N_t} \sum_{t} \left( f_P(\lambda,t) -\overline{f_P}(\lambda)\right)^2,
\end{equation}
%MJK added this
hence equalling to the rms value of $f'_P$ (without taking the square root). 
%MJK

\item min/max $\sigma_{\mbox{mm}}$: (blue shaded area): this shows the minimum and the maximum  of $f(\lambda,t) - \overline{f}(\lambda)$:
\begin{equation}
\sigma_{\mbox{mm}}(\lambda) = [ \min_t (f(\lambda,t)),  \max_t (f(\lambda,t)) ] -\overline{f}(\lambda)
\end{equation}

\item min/max (mean) $\overline{\sigma}_{\mbox{mm}}$: (green shaded area): this shows the mean minimum and maximum  of $f(\lambda,t) - \overline{f}(\lambda)$:
\begin{equation}
\overline{\sigma}_{\mbox{mm}}(\lambda) = [ \overline{ \min_t (f(\lambda,t)) }, \overline{ \max_t (f(\lambda,t)) }] -\overline{f}(\lambda)
\end{equation}

\item mean(sdev(patch)) $\overline{\sigma_P}$ (error bars): the standard deviation of the $f$-mode strength
%AL added this
($f(\lambda,t)$)
%AL
 is computed individually for all lat / time bins. This quantity  ($=\sigma_P(\lambda,t)$) is then averaged over time:
\begin{eqnarray}
\overline{\sigma_P}(\lambda) &=&  \frac{1}{N_t} \sum_{t}  \sigma_P(\lambda,t) \mbox{, with} \\
\sigma_P^2(\lambda,t) &=& \frac{1}{N_P}\sum_{\mbox{patch}} ( f(\lambda,t)
- f_P(\lambda,t) )^2,
\end{eqnarray} 
%AL added this
where $\sum_{\mbox{patch}}$ indicates the summation over all $f$-mode strengths within a patch, and $f_P(\lambda,t)$ being the mean $f$-mode strength in this patch. The patch size must be chosen small enough to not contain a significant variation in latitude or solar cycle. 
This quantity $\overline{\sigma_P}(\lambda)$ is displayed as the orange shaded area in \fig{butterfly_lat}.\\
%MJK Added this
This quantity is, therefore, a mean of the $f'_{P, {\rm rms}}$ (so this time also taking the square root) values in individual bins. The problem is that the quantity we are interested in inspecting, and defining the significance of, is now defining the error bar of the measurement. 
%MJK
%AL comment:
%My definition is maybe poor. Yes, I think it should be simply called the 'average rms fluctuation of the f-mode strength over 2 months' (in the above plot the patch size in time is 2 months).
\end{itemize}
\subsection*{All the stuff above this line not needed !!!}
\color{black}
%AL END unnecessary stuff

