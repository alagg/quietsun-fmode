%\documentclass[twocolumn,linenumbers,trackchanges]{tex/aastex631}
\documentclass{aa}
\usepackage{xcolor}
\usepackage{trackchanges}
\usepackage{gensymb}
\usepackage[colorlinks,allcolors=blue]{hyperref}
\usepackage{comment}
%\input{journal} 

\usepackage{graphicx}

\definecolor{cmt}{rgb}{0.5,0.0,0.0}
\definecolor{al}{rgb}{0.6,0.2,0.0}
\definecolor{mk}{rgb}{0.4,0.4,0.0}
\definecolor{no}{rgb}{0,0.0,0.4}
\definecolor{ns}{rgb}{0.4,0.0,0.0}
\definecolor{hr}{rgb}{0.4,0.0,0.0}

\newcommand{\al}[1]{{\color{al}AL: #1}}
\newcommand{\mk}[1]{{\color{mvn}MJK: #1}}
\newcommand{\no}[1]{{\color{corr}NO: #1}}
\newcommand{\ns}[1]{\textsc{\color{al}NS: #1}} 
\newcommand{\hr}[1]{\textbf{\color{ref}HR: #1}} 
\newcommand{\todo}[1]{{\color{cmt}ToDo: #1}}
\newcommand{\cmt}[1]{
{\color{cmt}#1}
}

%\renewcommand{\fig}[1]{Fig.~\ref{#1}} 
\newcommand{\fig}[1]{Fig.~\ref{#1}} 
\newcommand{\Fig}[1]{Figure~\ref{#1}} 
\newcommand{\eq}[1]{Eq.~\ref{#1}} 
\newcommand{\figs}[1]{Figs.~\ref{#1}} 
\newcommand{\tab}[1]{Tab.~\ref{#1}} 
\newcommand{\Tab}[1]{Table~\ref{#1}} 
\newcommand{\sect}[1]{Sec.~\ref{#1}} 
\newcommand{\Sect}[1]{Section~\ref{#1}} 
\newcommand{\secs}[1]{Sections~\ref{#1}} 
\newcommand{\app}[1]{App.~\ref{#1}} 


\newcommand{\kms}{km\,s$^{-1}$}
\newcommand{\ms}{m\,s$^{-1}$}
\newcommand{\grad}{$^\circ$}
\newcommand{\carcsec}{$\mbox{.\hspace{-0.5ex}}^{\prime\prime}$}
\newcommand{\halpha}{H$\alpha$}
\newcommand{\hminus}{H$^-$}
\newcommand{\hei}{He\,\textsc{i}}
\newcommand{\sii}{Si\,\textsc{i}}
\newcommand{\fei}{Fe\,\textsc{i}}
\newcommand{\tii}{Ti\,\textsc{i}}
\newcommand{\cai}{Ca\,\textsc{i}}
\newcommand{\caii}{Ca\,\textsc{ii}}
\newcommand{\caiih}{Ca\,\textsc{ii\,h}}
\newcommand{\coi}{Co\,\textsc{i}}
\newcommand{\helixp}{\textsc{HeLIx$^+$}}
\newcommand{\imax}{{IMaX}}
\newcommand{\sufi}{\textsc{SuFI}}
\newcommand{\sunrise}{\textsc{Sunrise}}
\newcommand{\los}{${\rm LOS}$}
\newcommand{\vlos}{$v_{\rm LOS}$}

\newcommand{\IN}{internetwork}
\newcommand{\inw}{\textsl{INw}}
\newcommand{\nw}{\textsl{Nw}}
\newcommand{\brms}{$B_{\rm RMS}$}

\newcommand{\colfig}[3][1.]{\begin{figure}\centering
    \includegraphics[width=#1\linewidth,clip=TRUE]{#2}
    \caption{#3}
    \label{#2}
\end{figure}}
\newcommand{\colfigtwocol}[3][1.]{\begin{figure*}\centering
    \includegraphics[width=#1\linewidth,clip=TRUE]{#2}
    \caption{#3}
    \label{#2}
\end{figure*}}


\graphicspath{{./figures/}}

\begin{document}
%\title{Quiet-Sun Magnetism Depends on Solar Activity}
\title{Solar-Cycle Variation of quiet-Sun Magnetism and Surface Gravity Oscillation Mode}
%\shorttitle{Quiet-Sun Solar Activity}
\titlerunning{Quiet-Sun Solar Activity}

%\shortauthors{Aalto-f-mode gang.}
\authorrunning{Aalto-f-mode gang.}


\author{N. Olspert\inst{1} \and H. Raichur\inst{2} \and A. Lagg\inst{3} \and M. K\"apyla\inst{1,3,4} \and H.-L. Truong\inst{1} \and N. K. Singh\inst{3}}
\institute{Department of Computer Science, Aalto University, PO Box 15400, FI-00076 Aalto, Finland \and
IUCAA, India \and
MPS, G\"ottingen, Germany \and
Nordita, Sweden
}



\abstract{The origin of the quiet Sun magnetism is under debate. Investigating the solar cycle variation observationally in more detail can give us clues about how to resolve the controversies.}
{We investigate the solar cycle variation of the most magnetically quiet regions and
their surface gravity oscillation ($f$-) mode integrated 
%power 
energy.
%as function of latitude.
}
{We use 12 years of HMI data and apply a stringent selection criteria, based on
spatial and temporal quietness, to avoid any influence of active regions (ARs).
We develop an automated high-throughput pipeline to go through all available magnetogram data
and to compute the surface gravity mode 
%power 
energy
for the selected quiet regions.}
{We observe a clear solar cycle dependence of the magnetic field strength in the most
quiet regions containing several supergranular cells. For patch sizes smaller than a supergranular
cell, no significant cycle dependence is detected. The \fff at the supergranular scale 
is not constant over time.
During the late ascending phase of Cycle 24 (SC24, 2011-2012), the \fff strength is roughly constant, but starts diminishing in 2013, as the maximum of SC24 is approached. This trend continues
until 2017, in the middle of which year we see hints of strengthening at higher southern latitudes.
Slow strengthening continues, stronger at higher latitudes than at the equatorial regions, but the \fff strength never returns back to the values seen in 2011-2012. Also, the strengthening trend continues past the solar minimum, to the years when SC25 is already clearly ascending. Hence the \fff behavior is not in phase with the solar cycle. 
}
{The solar cycle dependence at the supergranular scale is indicative for the fluctuating magnetic
field being replenished by tangling from the large-scale magnetic field, and not solely due to
the action of a fluctuation dynamo process in the surface regions. 
The absence of variation at smaller scales might be an effect of the
limited spatial resolution and magnetic sensitivity of HMI.
An anticorrelation of the \fff strength w.r.t. the solar cycle is expected, as active regions efficiently damp it. The \fff behavior, although showing such behavior in gross terms, is much more complex than this - in addition we see a phase shift, and different damping behavior of the ascending phases of SC24 and 25. We speculate that the latter two could be related to the magnetic helicity-constrained sub-surface toroidal field rather than instrumental effects. 
}

%MJK Added keywords
\keywords{Solar physics -- helioseismology, solar cycle; Solar activity --- solar active regions; Solar magnetic fields --- Solar dynamo}

\maketitle

\section{Introduction} \label{sec:intro}

%NS: adding some text which can be moved around later to a more suitable place
%NS: please modify at wish, here and elsewhere
Localized regions of intense bipolar magnetic structures, called active regions
(ARs), are seen on the solar surface. Their numbers vary periodically and trace the
butterfly diagrams which exhibit a cyclic magnetic activity of the Sun in a latitude--time
domain. Such diagrams have proven to be useful and reveal some properties of solar
magnetism, the origin of which is not yet fully understood.
%Due to a number of reasons, it would be 
%extremely useful to identify those localized regions
%in advance where ARs are going to emerge later. 
In an observational study involving
HMI data, \cite{SRB16} reported strengthening of solar surface or the fundamental \fff
about one to two days before
the formation of AR on the same corotating patch of area $(180\, \Mm)^2$,
%MJK
and quenching of the mode after the formation of the AR has been postulated
(REFS) and reported (REFS).
%AL Is it common to refer to arxiv-only publications? We know that it was rejected.
%MJK Yes, it is common enough.
Very recently, another study reported similar results using a different
helioseismic technique (Fourier-Hankel method, \cite{Waidele22}).
%MJK
%Detections of such localized perturbations in the solar \fff due to ARs can have
%important applications for early forecasting of the space weather. 
%These 
Detection of such localized perturbations
could also shed light
on the physics behind the formation of ARs.
%MJK
Furthermore, such behavior of the \fff has been seen in compressible
isothermal MHD simulations (REFS). 

%MJK
The observational studies of \cite{SRB16} and \cite{Waidele22} used quiet Sun (QS)
regions on the opposite hemisphere to compare with the AR \fff. This method
requires that a quiet patch exists on the other hemisphere, and hence 
limits the number of ARs that can be included in the hindcasting procedure. It is
also prone to be affected by the probable fluctuations in the QS
\fff level. Although the results look promising, proper calibration with
statistically sound QS level, not just a comparison of a random QS patch on the
other hemisphere, is necessary to prove the robustness of these findings. Also,
such a calibration procedure is necessary for increasing the sample size. 
Building such a QS calibration data product is one of the main aims of this
study: we carefully identify the quietest regions on the solar surface based on the level of magnetic activity observed in
line-of-sight (LOS) magnetograms that are readily available from HMI, and
compute the \fff  strength at the central meridian as function of latitude
and time with suitable averaging. 
%MJK

%MJK 
Building such a data product for the \fff, requiring us to identify the most
inactive regions on the solar surface, allows us to extract statistics
of the QS magnetism as well. 
Although QS magnetism is weak, HMI provides high sensitivity, high spatial
resolution, long-term stability, and constant conditions.
It has been argued that the QS magnetic field is 
independent of the solar cycle 
%AL added some references
\cite[]{Kleint+10,Buehler+13,Faurobert+15,Jin+15a,Jin+15b} 
and some other studies have proposed
that some dependence should exist \cite[]{Lites+14,Faurobert+21}. 
%AL
There are two sources of magnetic
fluctuations: the small-scale dynamo (SSD) instability and the action of turbulent
convective motions on the underlying large-scale magnetic field (usually
called tangling, REFS). These processes are inseparable and without a clear boundary, 
as the turbulent driving of the SSD and tangling most likely occur at similar
scales, akin to the convective turbulence itself being the driver of both
effects. These effects, however, might have a different timescale, and hence
be distinguishable: SSD is prone to replenish the magnetic fluctuations 
exponentially, while the tangling can be expected to be linear in time.  
Hence, we could expect a scenario, where at the smallest scales, the magnetic
field would be only be replenished rapidly by the SSD, while the larger
convective cell boundaries would accumulate magnetic field from both the
 effects due to flux expulsion (REFS). Hence, one would expect no solar cycle
 dependence for the interiors of the convective cells, while a dependence
  should be observed for the magnetic fields at scales that include also
  the convective cell boundaries. We also set out to investigate these
  possible scenarios with our data products. 
%MJK

%MJK
The paper is organized as follows: in Sect.\ref{pipeline} we describe
the data, the necessary steps to 
%prune 
clean it, and the automated pipeline we built for harvesting the data and compiling the end products, namely the QS magnetism data products, and the QS and AR \fff data. In
Sect.\ref{results} we discuss our findings for the QS magnetism, \fff, and the lastly present some AR data with the QS calibration
applied. 
%MJK

%Nature of AR induced \fff perturbations appear complex with respect to a level
%corresponding to the quiet Sun. There is first an enhancement of
%\fff power about 1-2 days before the emergence, followed by a damping of the mode
%as the AR begins to emerge; see \cite{SRB16}, for more details.
%This makes it harder to predict a newly forming ARs in close proximity to existing ARs
%which would have already caused the damping of \fff in such a `crowded' environment.
%Even in the quiet phase of the Sun, local \fff power displays a systematic variation
%depending on the location of the patch on the solar disk. Its power decreases as we move away
%from the disk center --- an effect which may be largely attributed to the limb-darkening.
%AL I'd rather say it's an effect of foreshortening: 
%AL: The amplitude of the oscillations in the LOS component decreases.
%\cite{SRB16} suggested the following fitting function to account for this variation
%in the \fff power during the quiet phase of the Sun:
%$\zeta(\cos \alpha)=\cos \alpha [q+(1-q)\cos \alpha]$ with $q=0.5$,
%AL We updated this formula a bit after the quiet Sun calibration to account mainly for effects
%AL at high alpha. But we do not use this calibration in this paper, maybe we can skip this eq.
%where the angular distance $\alpha$ from the disk center can be expressed in terms of
%latitude $(\vartheta)$ and longitude $(\varphi)$ of point of interest as
%$\cos \alpha = \cos \vartheta \cos \varphi$.
%NS.

%NS: continued
%It would be extremely useful to understand the background evolution of quiet-Sun \fff
%over solar cycles in order to more reliably predict the photospheric emergence of ARs,
%regardless of their environment, crowded or isolated. This is expected to enable us explore
%a statistically large number of ARs by studying the properties and evolution of their
%associated local \fff power. This provides a good motivation to determine a butterfly
%diagram for the whole cycle, as mentioned above, but now based on the solar \fff power
%from the magnetically quiet patches. Present work aims to address this by carefully identifying
%the quietest regions on the solar surface based on the level of magnetic activity %observed in
%line-of-sight (LOS) magnetograms that are readily available from HMI. 
%NS.

%MJK We would need some text about the magnetic helicity business
%MJK also to the intro.

%\begin{itemize}
%\item quiet Sun magnetism is weak. Measurement requires high sensitivity, high spatial %resolution, otherwise signal cancellation. Variations in quiet Sun magnetism are even %tougher to detect. Require long-term stability, constant conditions, HMI offers this, %now 1 solar cycle in orbit. 
%\item what is the quiet Sun? simple: No AR. 98\% of the solar surface are in this %state. But even the quiet regions are structured: network / internetwork. 
%\item what is expected to vary with solar cycle? network, IN, both, or nothing? A few %words about the expectations.
%\item argue why taking into account network is important: the IN flux is advected %towards the network boundaries. A cycle variation of the IN flux therefore should be %reflected in a variation of the network flux. Network accumulates IN flux and %therefore acts as a memory of what happened in the IN. Helps to overcome detection of %the weak IN fields (difficult enough), even more difficult to study variations of %these weak fields.
%\item HMI: compared to Hinode / ground based / Sunrise: not the most sensitive %instrument to B. But: Stability allows averaging. 8 hr time and spatial averaging %allows to boost S/N ratio (can we give a number here?)%
%
%\end{itemize}
%
%\cite[]{2019LRSP...16....1B}
%Luis Talk PHI meeting: 38\% of the flux emerging in the IN makes it to the network. %Network flux is constant, the IN flux to the network accumulates quickly to network %flux.
%
%\cite[]{2015ApJ...806..174J}
%
%\cite[]{2013A&A...555A..33B}
%
%\cite[]{2021arXiv210508657F}
%
%\cite[]{2021arXiv210514533R}
%
%\cite[]{ballot2021changes}


\section{Observations}\label{pipeline}

Our analysis is based on data from the Helioseismic Magnetograph and Imager \cite[HMI,][]{2012SoPh..275..207S,2012SoPh..275..229S} on board the Solar Dynamics Observatory \cite[SDO,][]{2012SoPh..275....3P}. We use two standard data products: (i) full-disk line-of-sight (LOS) magnetograms, computed every 720\,s by combining filtergrams obtained over a time interval of 1260\,s (\texttt{hmi.M\_720s}), and (ii) full-disk LOS dopplergrams, computed every 45\,s from six positions across the nominal 6173.3\,\AA{} spectral line (\texttt{hmi.V\_45s}). We processed the two data sets in a semi-automatic pipeline (see \fig{pipeline}), optimized for obtaining reliable information about the magnetic field in the QS regions and for a robust computation of the \fff power from the dopplergrams. The left tree in \fig{pipeline} describes the pipeline used for the magnetograms, the right tree for the dopplergrams.
In the following we use the following notation and definitions. 
We denote the solar 
latitude with $\lambda$, longitude with $\phi$, 
both in the Stonyhurst coordinate system \cite[]{Thomson06}, 
and time with $t$. 


\colfig{figures/Pipeline2}{Data pipeline. The leftmost path illustrates the pipeline to collect the magnetograms in the most quiet patches, while the rightmost one the pipeline to track data from the quiet patches from the HMI database, and compute the \fff energy. The central path of AR processing is otherwise equivalent to the QS pipeline, but there AR coordinates are sent for tracking in the German cluster environment, and an additional step, namely QS calibration, is performed in the end. Rectangular boxes represent analysis functions, and ellipsoids the derived data products. The rectangles with yellow frames stand for the corrective functions applied to the data. }

\subsection{Magnetograms}

The first data product, full-disk line-of-sight (LOS) magnetograms, provides a direct measurement of the variability of the %quiet-sun 
QS
magnetism during a solar cycle. To enhance the signal-to-noise ratio in the LOS magnetograms we performed a newly developed algorithm for spatial and temporal averaging: The full-disk HMI magnetograms starting from 27-Apr-2010 and ending at 
%MJK complete todo item.
\todo{03-May-2022} 
were downloaded from the Joint Science Operation Center (JSOC) hosted at Stanford University (http://jsoc.stanford.edu) to a temporary storage (see \fig{pipeline}, 'mahti storage') and tracked at full spatial resolution for 8 hours to compensate for the solar rotation ('full disk tracking').
%AL @Nigul: Can you explain the differential rotation model used?
%NO As much as I remember we omitted taking into account diff. rot. as it was marginal during 8hrs
%MJK Then we must say it.
We neglected differential rotation due to
the short tracking time.
The step between tracked sequences was 4 hours, so that a total of 6 tracked sequences were gathered per one day, resulting in more than 
%MJK Check.
\todo{$31\,000$} tracked sequences.

%MJK Latitude and longitude have now been defined, should use them.
From each tracked sequence, we extracted two data products by dividing the visible solar disk  between latitudes and longitudes from $-80\degree$ to +$80\degree$ into (i) $64\times 64$  overlapping patches of $15\degree$ (in solar latitude and longitude)  and (ii) $180\times 180$ patches of $1\degree$.
Every of these patches therefore contains a space-time cube of LOS magnetograms at full spatial resolution at a 12-minute cadence, %allowing 
from which we 
%AL
compute the 
%have chosen to compute the following statistics:
%to compute the statistical properties (
%mean, standard deviation, skewness, and kurtosis.
%) 
%of the 
%The following parameters: 
%MJK Definition of the mean clashes with the butterflies, as this is now spatial and temporal mean, while later it is used for averages over bins.
%MJK B LOS is basically a vector, hence should we rather introduce a boldfaced quantity?
%AL not needed:
%$\langle B\rangle$ -- mean of the magnetic field strength, $\langle|B|\rangle$ -- mean of the absolute value of the magnetic field strength, and %$\sqrt{<B^2>}$
%AL 
root-mean-square of the magnetic field strength, averaged over the full field-of-view of the tracked cube (\brms{}=$\sqrt{\langle B^2\rangle_{\rm LOS}}$).
%(\brms{}). 
The $15\degree$ patches (i) are large enough to cover several supergranulation cells containing network and \IN{} fields \cite[]{2010LRSP....7....2R} with a typical size of 30--35\,Mm (we refer to them as \nw{} cubes), and the $1\degree$ patches (ii) are small enough that some of them lie completely in the \IN{} (\inw{} cubes). 
The statistics for each patch were stored as data products (see also \fig{pipeline}) including the information about 
%MJK Different coordinate systems may be confusing.
latitudinal and longitudinal position as well as the Carrington %MJK Starting point, central point,...?
longitude for network and \IN{}. We refer to these maps as the \nw{} and the \inw{} statistical maps.

 
% \begin{itemize}
% 	\item $\mu_B$ -- Mean of the magnetic field strength
% 	\item $\sigma_B$ -- Standard deviation of the magnetic field strength
% 	\item ${\rm skew}_B$ -- Skewness of the magnetic field strength
% 	\item ${\rm kurt}_B$ -- Kurtosis of the magnetic field strength
% 	\item $\mu_{|B|}$ -- Mean of the absolute value of the magnetic field strength
%     \item $\sigma_{|B|}$ --Standard deviation of the absolute value of the magnetic field strength
%     \item ${\rm skew}_{|B|}$ -- Skewness of the absolute value of the magnetic field strength
%     \item ${\rm kurt}_{|B|}$ --Kurtosis of the absolute value of the magnetic field strength
% 	\item \brms{} -- Root mean square of the magnetic field strength
%     \item $\sqrt{\sigma:{B^2}}$ -- Root mean square of the standard deviation of the magnetic field strength
%     \item $\sqrt{{\rm skew}_{B^2}}$ -- Root mean square of the skewness of the magnetic field strength
%     \item $\sqrt{{\rm kurt}_{B^2}}$ -- Root mean square of the kurtosis of the magnetic field strength
% \end{itemize}

\subsection{Quiet region selection\label{quietregion}}

From the 
%parameters 
statistics computed from the cubes, 
%the root mean square of the magnetic field strength (
\brms{}
%) 
turned out to be the best tracer for determining the magnetic activity level. It could clearly distinguish between $15\degree$ patches containing active regions, plage, enhanced network and quiet network. Also, it depicted very well the low-field \IN{} regions.

The analysis of the solar-cycle variation of the 
%quiet-sun 
QS
magnetism required a careful selection of the most quiet regions, defined as being free of enhanced solar activity. We therefore searched for the minimum value of \brms{} in both, the \nw{} and the \inw{} statistical maps, on a latitudinal grid with a $10\degree$ spacing fulfilling the following additional criteria:
\begin{enumerate}[(i)]
	\item\label{c1} the most quiet pixel must be within $\pm10\degree$ around the central meridian,
	\item\label{c2}  this pixel must belong to the 50\% most quiet pixels of the month,
	\item\label{c3}  this pixel must be the most quiet pixel within a 4-day interval.
\end{enumerate}

Criterion (\ref{c1}) was chosen to get the strongest magnetic field signal along the central meridian. (\ref{c2}) guarantees an equal distribution of quiet pixels over the 11-year period of available HMI measurements, and (\ref{c3}) ensures that the quiet pixels for the 1-month period do not originate from the same supergranular structure, since the dynamical evolution time of the supergranulation lies between 24 and 48\,h \cite[]{2010LRSP....7....2R}. The result of this selection were two time series of the \brms{} for the most quiet patches in the network and the \IN{} regions. Note that the this selection also efficiently removes the 24\,h modulation present in the HMI magnetograms.

\subsection{Correction for HMI sensitivity change\label{sensicorr}}

The 11-year temporal evolution of \inw{} time series {the \brms{}} value revealed clearly a change in the HMI observing mode, performed on 13-Apr-2016. On this day, HMI switched to a more efficient observing mode \cite[see][]{2018SoPh..293...45H,2014SoPh..289.3483H,2016SoPh..291.1887C}. By combining both HMI cameras to determine the vector-field observables the cadence for full-disk magnetograms could be reduced from 135\,s (observational mode MOD-C) to 90\,s (MOD-L). This reduced the noise level for Stokes $V$ measurements by 17\%, resulting in a decrease of the noise level in the \los{} magnetograms by 5\%.

For the long-term study presented in this paper, we need to correct for this sensitivity change. A very accurate correction method can be derived from the \inw{} time series: since it contains only the most quiet pixels over a certain latitude region and time, the sensitivity change results in a step function. The value of the step was determined by fitting a polynomial to the \brms{} values determined from the \inw{} time series plus a Heaviside step function, centered at the date of the mode change. We used the Bayesian information criterion \cite[BIC,][]{Stoica2004} to determine the degree of the polynomial, which lies between 1 and 6 for the various latitudes. We want to note that the retrieved amplitude of the Heaviside step function is only weakly dependent on the degree of the polynomial. This fitting is exemplified in \fig{fit-offset} for the solar latitude 0$^\circ$, where the minimum value for the BIC was reached for a fit with a polynomial of degree 4. The so-determined amplitude of the Heaviside step function is added to the data points after 13-Apr-2016 for all data presented in this paper.



%MJK Remove the title; typeset y-axis in latex format; insert x-axis label
%MJK RMS we usually write with lower case, hence unify RMS -> rms
%MJK throughout
\colfig{fit-offset}{Determination of the correction for the HMI sensitivity change: the observing mode change on April 13 2016 causes a discontinuity in the level of \brms{} values of the \IN{} data. The original data are displayed in with the dark red and black dots, the corrected data with the light red dots. The dashed line indicates the polynomial fit of degree 4 used to obtain the offset.}




% \cmt{
% Key issue: 
% How to obtain the most reliable HMI data product telling us about the variability of the quiet Sun magnetism during a solar cycle? Idea: combine spatial and temporal averaging, and analyze the statistics in space-time cubes to determine the level of quietness as a function of latitude. This requires the magnetic field information coming from specro-polarimetry. 

% But first the boring stuff: How to obtain a clean, 11-year long data set:

% About the long term trend: The jump in the data happened on 13 April 2016 (search for this date in \cite{2018SoPh..293...45H}). There it states:
% \begin{itemize}
% \item Standard HMI observations were initially obtained with a framelist called Mod-C that
% repeated every 135 seconds. Mod-L, a 90-second FTS, replaced Mod-C on 13 April 2016.
% The two versions of Mod-C have FTS ID 1001 or 1021; the Mod-L HFTSACID is 1022.
% Some calibration framelists changed when the standard sequences changed.
% \item Since 13 April
% 2016, filtergrams from the two cameras have been combined to compute the vector magnetic
% field \cite[]{2014SoPh..289.3483H,2016SoPh..291.1887C}
% \item On 13 April 2016, after the prime mission ended, HMI switched to
% a faster sequence, FTS ID 1022, also known as Mod-L. The Mod-L sequence requires that
% images from both cameras be combined to determine the vector-field observables. 
% \end{itemize}
% Mod-L description here: http://hmi.stanford.edu/hminuggets/?p=1596: \textsl{
% Mod-L provides all of the filtergrams necessary to compute the Stokes parameters [I, Q, U, V] in 90 seconds, instead of the 135 seconds required for Mod-C. Thus Mod-L increases the maximum temporal resolution for measuring full Stokes parameters. At the same time, it decreases the noise because twice as many filtergrams are available. The 45s data products from the front camera are also improved; since there is no longer a 135s period in the instrument configuration, the corresponding peak in the Doppler power spectrum is now gone. The new 90s period is at the Nyquist frequency of the 45s-cadence Doppler data. The Stokes measurement is normally averaged over time (nominally 720 seconds) to derive [I, Q, U, V], and we now combine three times as many CP and 1.5 times as many LP measurements. Table 1 compares the Mod-C and Mod-L observations.}
% }



\subsection{Dopplergrams}

The second HMI data product used in this paper are the \los{} velocity maps. The goal is to compute the 
%power 
energy contained in the
%of the 
surface gravity mode, the so-called \fff,
%power, 
%MJK This strikes me as odd.
%as an independent measure to quantify the solar cycle variation.
and investigate whether variability during the solar cycle is present. Such variations could be caused by the presence of non-emerging sub-surface magnetic fields, as the \fff is known
to be strongly affected by the presence of magnetic fields (REFS). 
It is also very important to study this question in more detail, for establishing a reliable calibration method to study the claimed 
\fff enhancement prior to AR emergence \cite{SRB16,Waidele22}. 

The dopplergram data is hosted in the German Data Center for SDO (GDC-SDO) on a server at the Max Planck Institute for Solar System Research (MPS Göttingen, Germany) whereas the analysis is executed in the CSC supercomputing environment (Aalto, Finland). We have utilized  a function-as-a-service client based on funcX \cite[]{chard20funcx} for accessing the required data in the database server. We have developed functions based on funcX API and deployed the functions in the MPS environment. These functions leverage the mtrack\footnote{\url{http://hmi.stanford.edu/teams/rings/modules/mtrack/v25.html}} command to prepare the dopplergram cubes within the GDC-SDO environment and subsequently transfer them to the CSC environment. The coordinates of the selected 
%quiet-sun 
QS
regions were sent to the funcX service, which invokes suitable functions to automate the data retrieval and movement.
%{the using remote function call service funcX \cite{chard20funcx} to MPS server where the dopplergram database is located. Next, using mtrack command the tracked dopplergram cubes were retrieved and sent back to CSC supercomputing environment.}

%AL: TODO details of the tracking: Central coordinate from quiet region selection, Harsha's parameters. $300\times 300$ pixels, 8 hrs ... 

%MJK To be decided how we refer to the final number obtained.
%MJK Nishant16 called that \fff energy, and used E_f for it.
%MJK As we are practically computing the same number, then 
%MJK we should possibly stick to using the same notation.
%MJK Main data product is the normalized E_f, for which
%MJK we could continue using the tilded quantity.
%
%MJK The lower panel should have P(omega,k) on the y-axis
%MJK Here the background is already subtracted? 
%MJK Explanation of its removal was not yet included, now
%MJK added something preliminary. Should we also show the
%MJK non-subtracted spectrum? 
\subsection{Computation of the \fff 
%power}
energy}

%In the CSC environment, 
%MJK
Subsequent processing involved calculating the 
%3d 
three-dimensional
power spectra for each dopplergram cube and 
%MJK Not averaging, but integrating, right?
%averaging 
integrating
the spectra azimuthally in $k_x, k_y$ plane for each constant frequency $\omega$,
%MJK
to obtain a collapsed power spectrum, $P(\omega,k)$. An illustrative case for a frequency
value at which \fff is strong, is shown in 
%(see 
\fig{ring_diagram},
%MJK
upper panel.
%MJK
Such 
%averaging 
a procedure
significantly reduced the noise level leading to smooth one-dimensional $k-\omega$ spectra,
%. 
%MJK
an example being shown in the lower panel of \fig{ring_diagram};
in this kind of a plot, the \fff is the rightmost peak at the highest $k$ values.
%MJK
%Such 
%The
%averaging 
%performed
The adopted procedure
is justified for the 
%quiet-sun 
QS
spectra, as the ring diagrams are radially symmetrical w.r.t. $\omega$ axis. 
%MJK Reforms, easily does not tell anything.
%From the obtained spectra \fff was
%easily separable from the rest of the modes and integration done over $k$ and $\omega$, to obtain full power of the \fff.
To obtain the total energy contained in the \fff, $E_f$, we
performed another integration over the separated \fff signal.
In contrast to earlier studies \citep{SRB16,Waidele22}, we
did not perform fitting to the \fff for its extraction. 
Instead, we only fitted the background, and subtracted this 
contribution.
Then,
an integration range in $k$ space was selected for each constant $\omega$ in a following way: first the maximum of \fff $k_{\rm max}$ and the minimum between \fff and first $p$-mode $k_{\rm start}$ were detected; $k_{\rm start}$ was chosen as the start of integration range and the end of the integration range $k_{\rm end}$ was set as $k_{\rm end}=k_{\rm max}+2(k_{\rm max}-k_{\rm start})$; $k_{\rm end}$ chosen in such a way guarantees that the integration range is sufficiently wide to cover significant part of the \fff 
%power 
signal in $k$-space.
Integration range in $\omega$ space was chosen between values starting from 
%MJK Should use some normalized numbers, and these do not match with the nu of the figure.
14.45 and ending at 29.03, 
%MJK
where the significant part of the \fff power resides. 
%MJK 
As the end result, we have computed the \fff energy
%MJK Must be beautified
\begin{equation}
E_f=
\sum_{\mbox{$k_{\rm start}$}}^{\mbox{$k_{\rm end}$}}
\sum_{\mbox{$\omega_{\rm min}$}}^{\mbox{$\omega_{\rm max}$}} P(\omega,k)
\end{equation}
After completing the described steps in the CSC supercomputing
environment, we have collected a total of \todo{18327 number to be updated for final version!} \fff 
%powers 
energies
for patches close to central meridian over all latitudes covering 
%the full solar cycle.
most of Cycle 24 and the ascending phase of Cycle 25.

%MJK The figure and caption still say "nu" for frequency. I propose
%MJK we revert to omega.
%MJK Also, the resolution of the upper panel is poor.
\begin{figure}\centering
	\includegraphics[width=1.0\linewidth,trim={0cm 0.4cm 0cm 0.3cm},clip=TRUE]{ring_diagram}\\
	\includegraphics[width=1.0\linewidth,trim={0cm 0cm 0cm 1cm},clip=TRUE]{az_avg_spec}
	\caption{Top: example of quiet-sun ring diagram at $\nu=3.56$. Bottom: spectrum obtained by azimuthally averaging the ring diagram. The red vertical line marks the position of the maximum of the \fff and the black vertical lines the range of integration.}
	\label{ring_diagram}
\end{figure}

\subsubsection*{Orbital correction of the \fff 
%power
energy}

%AL We have to be consistent. Do we call it \fff power, strength or area?
%NS: strength is perhaps better; we could mention somewhere the strength and power
%NS: are being used interchangeably, if needed
Since the \fff 
%power 
energy
is computed from the \los{} dopplergrams, its value depends strongly on the viewing geometry. This dependence,  roughly following the cosine of the solar latitude for data taken at the central meridian, is additionally modulated by the orbital motion of the Earth around the Sun, which changes the viewing angle at any given solar latitude by $\approx\pm7^\circ$ during one year. We compensated for this periodic variation by fitting the parameters ($x_0(\lambda), ..., x_3(\lambda)$) of the following function to all observations of a given latitude $\lambda$:
\begin{eqnarray}
\label{eq:orbitcorr}
A_{\mbox{corr}}(\lambda) &=& x_0(\lambda) (  2 ( (1+\cos(\alpha+x_1(\lambda)))/2)^{x_3(\lambda)}-1   )\nonumber \\
&+&x_2(\lambda),
\end{eqnarray}
with $\alpha$ being the phase angle of the Earth defined as the cotangens computed from the $x,y$ barycentric position of the Earth.
This correction $A_{\mbox{corr}}(\lambda)$ is then subtracted from the computed 
%\fff power
$E_f$ values, which efficiently removes any yearly variation. 


%AL Mention also the focus change in 2018-10-16, leading to a change in the \los{} maps.

\section{Results}\label{results}

After applying the selection criteria described in \sect{quietregion} and the subsequent correction for the 
%MJK
%HMIS 
HMI's
sensitivity change (\sect{sensicorr}) we obtain the dependency of the \brms{} values of the most quiet region at the central longitude as a function of heliographic latitude and time. \fig{Brms-lat00-percentile} presents the \brms{} values from May 2010 until October 2021 for the heliographic latitude $\lambda = 0\degree$. The individual data points (red dots) are computed over an area of 15$^\circ$ in latitude and longitude and over a time of 8 hours. At disk center, this corresponds to an area of $\approx 180 \times 180$\,Mm$^2$, and therefore contains 30--40 supergranular cells. The \brms{} value 
%therfore contans 
therefore contains
network and \IN{} fields.
Typical magnetograms at disk center at solar minimum (July 2010) and maximum (Nov 2014) are presented in \fig{2010-07-21_2014-11-09}. There is no obvious difference between the two maps: both show similar minimum and maximum field strengths, and the visible impression of the supergranular cells with the strong field patches of both polarities surrounding the \IN{} regions is indistinguishable.

\fig{Brms-lat00-percentile} clearly indicates the solar cycle dependence of the 
%quiet-sun 
QS
magnetic fields, consisting of network and \IN{}. The \brms{} values peak in the second half of 2014, close to the declared maximum of 
%MJK The peaks appear later than that to me...
solar cycle \#24 (April 2014, \todo{REF!}. The variation is statistically significant, as indicated by the the mode and the 97\% percentiles, computed by fitting a log-normal distribution over a 1-year sliding window.

A similar plot is presented in \fig{Brms-lat00-percentile-IN}, but now the \brms{} value was computed only for a $1\degree$ window in latitude and longitude, corresponding to an area of only $12\times 12$\,Mm$^2$ at disk center. The quiet-region selection criteria (\sect{quietregion}) guarantees, that this patch lies fully within the \IN{}, and is not `contaminated' by network fields. Clearly, there is no dependence of \brms{} with solar cycle detectable. The 
%MJK It is not self-evident how decrease in sensitivity due to aging would lead to detection of higher mgfs.
slight increase over the 11-year period might be an effect of the aging of the HMI instrument and the resulting decrease in the sensitivity for magnetic field measurements.

The average \brms{} value in the \IN{} is in $\approx6$\,G. Using the area of one HMI pixel this corresponds to a magnetic flux of $3\times 10^{16}$\,Mx. 
%MJK To be compared with some earlier values/estimates?

\colfigtwocol{2010-07-21_2014-11-09}{Comparison of two disk-center magnetograms at solar minimum (2010-07-21) and 
%MJK
around the maximum seen in the disk-center \brms{} values
%solar maximum 
%MJK But this is not the official maximum based an sunspot number?
%MJK Now changed the text above, but the main text should perhaps also be modified.
(2014-11-09). The maps show the first frame of the 8\,h data cube, the animation is in the online material.}


%MJK Axis labels
\colfig{Brms-lat00-percentile}{\brms{} determined from the most quiet patches from 2010 to 
%2021 
2022
at disk center. The patches of 15$^\circ$ size in longitude and latitude contain network and \IN{} fields. The solid red lines display the 97\% percentile level, the dashed, gray line the mode of a log-normal fit to yearly-binned data moved in a sliding window of 100 days length.
%MJK
For all values of $\omega$ the magnitude of $P$ is always an order of magnitude higher than
the noise level outside the considered location of the ring.
%MJK
}


%MJK Axis labels
\colfig{Brms-lat00-percentile-IN}{Same as \fig{Brms-lat00-percentile}, but for \brms{} determined from patch size of only of 1$^\circ$ in longitude and latitude, corresponding to \IN{} patches. }


%\subsection{Butterfly}
\subsection{Quiet Sun \fff butterfly diagram}\label{qsf}


%MJK Basic stuff to be moved upwards and used throughout.
%Definitions:\\
%$\lambda$ = latitude, $t$=time, \\
%MJK
%MJK
To investigate whether the \fff energy has any significant solar cycle dependence, we construct butterfly-diagram-like plots by computing and removing the temporal mean, and plotting the so-obtained data in a binned latitude-time diagram. 
We compute the temporal mean over the whole time series as 
\begin{equation}
    \overline{\mef}(\lambda) = 1/n_t \sum_t \mef(\lambda,t),
\end{equation}
where $n_t$ is the total number of time points at a certain $\lambda$, and denote it with an overbar. Next, we compute the variation of the \fff energy around this level, defined as
\begin{equation}
    \mef'(\lambda,t)=\mef(\lambda,t)-\overline{\mef}(\lambda),
\end{equation}
and plot this quantity as function of latitude and time as in \Fig{butterfly_lat}, with a 3-month binning in time and roughly 6 degrees in latitude. 

To estimate 
%MJK Now removing the noise concept at least from here.
%whether this variation is significant w.r.t. noise, 
the overall variability level of the QS, and compare it to the cycle variation level defined above,
we will use the standard deviation of the data in a patch.
We take $N_t$ bins in time, $N_\lambda$ in latitude, and denote the number of \ef measurements contained in such a patch by $N_P$. We compute the average \fff energy in a patch as
\begin{equation}
\langle \mef \rangle=1/N_P \sum_{N_P} \mef (\lambda,t).
\end{equation}
Then we can straightforwardly calculate the standard deviation as
\begin{equation}
\sigma= \sqrt{\frac{1}{N_P}\sum_{\mbox{patch}} ( \mef(\lambda,t)
- \left<{\mef} \right>)^2}.
\end{equation}
From this we form the average error at each latitude by taking an average over time and denote as $\overline{\sigma}$. 
%AL added
This quantity is plotted as the error bar in the right panel of \fig{butterfly_lat} and similar plots.
%MJK This still holds
By inspecting the level of variability around the temporal mean and the 
%mean error we must conclude that the signal is above the noise level, 
overall variability of the QS, we must conclude that these signals are of the same order of magnitude. 
In this respect, the solar cycle dependent signal is very weak, but if we compare that to the noise level in the data, both the QS variability level and cycle-dependent signal exceed the noise level by an order of magnitude, hence this is not an issue concerned with the instrument sensitivity.
%Hence, the signal is very weak, but it is still meaningful to proceed with the analysis.
%MJK This part is not very convincing yet.

What could be the cause of the strong QS variability level is most likely related to the strongly varying sub-surface magnetic fields that affect the \fff that cannot be removed by using the surface \brms{} as the selection criterion for really quiet patches. This is illustrated in \Fig{fmode_vs_brms}, where we show the measured \fff energy versus the \brms{} in the patches. The \fff energy is not correlated with the \brms{} seen at the surface, and strong, nearly constant, variability is seen at different epochs of the solar cycle. This poses clear limitations of the calibration method proposed here to detect weak transient signals in the \fff evolution.




%AL create this plot with:
% %run plot_fmode_butterfly.py -i ../qs_cubes --reread=False --fov 'FFOV' --show_cos=False --show_cycle=False
\colfigtwocol{butterfly_lat}{Butterfly diagram of \fff %strength 
energy variations, $E'_f(\lambda,t)$, computed from azimuthally integrated 
%average of the 
ring diagrams, are shown in the left panel. 
%The 
%latitudinal average over time, 
%displayed in the right panel, was removed for better %visibility of the variation. 
The temporal average, $\overline{\mef}(\lambda)$, that was subtracted from the data, is shown as the black line in the right panel. 
Its error bars
%, plotted on the latitudinal average of the \fff strength, 
represent the time averaged standard deviation of the fluctuations of the \fff 
%strength 
energy
around the signal for every bin ($\overline{\sigma (\lambda)}$). 
%The orange shaded area shows the amplitude of the solar cycle variation ($f'_A(\lambda)$). 
}

\colfig{butterfly_lat_cut}{Cut through butterfly diagram of \fff energy variation for three different latitudes (see legend). The solid line represent a fit using a sine function.}

%The horizontal lines in the colorbar indicate $\pm \overline{\sigma_P}(\lambda)$.


\Fig{butterfly_lat} displays the butterfly diagram of the QS
\fff energy determined
from $15\degree \times 15\degree$ patches that showed only weak, small-scale magnetic structures.
To compute the \fff 
energy, we have used the whole ring diagram data.
Temporally averaged profile in latitude is removed for highlighting the variation in the
\fff power. The period covered is somewhat larger than the 11 year solar cycle~24, which
began to show some magnetic activity from early 2010; this cycle ended by around 2020.
Broadly, we find an anti-correlation of the \fff 
energy with the activity cycle of the
Sun --- during the solar minimum and early rising phase ($\sim 2010-12$), the \fff 
energies
are larger, and around cycle maximum\footnote{the Sun reached its peak magnetic activity
in terms of number of sunspots during the cycle~24 in April 2014.} to the declining phase,
the \fff 
energy 
is suppressed.
This correlation is strongest near the solar equator, within about $\pm 20\degree$ in
latitude, and it decreases with higher latitudes.
Interestingly, we observe a temporal shift in the \fff butterfly diagram with respect to the
solar cycle, such that the minimum of the \fff occurs later, i.e., in the declining phase of
SC24, roughly around the end of the year 2016.


%MJK New para; describing the rest of our observations.
During mid-2017, we see weak strengthening of the \fff strength to commence. During the minimum (2018-2019) between the Cycles 24 and 25, the strengthening appears stronger in the higher latitudes. It also continues after the minimum, when Cycle 25 has already started its ascending phase. What is also noteworthy is that the \fff 
%strength 
energy
never rises as high as the values observed during the ascending phase of Cycle 24 (2011-2012).
%MJK

\subsection{AR \fff with QS calibration}\label{arf}

%MJK Added
In this section we compare the \fff time evolution before and after the emergence of 
two ARs (11130 in \Fig{AR11130_track} and AR11105 in \Fig{AR11105_track}), that were reported to show an enhancement both in \cite{SRB16} and \cite{Waidele22} w.r.t. a QS control patch
in the opposite hemisphere several days before their emergence. The main difference in our analysis
is that the \fff energy is now normalized to the actually measured average QS level
around the specific time and latitude. To minimize
the QS fluctuations, Gaussian smoothing to the QS \ef\,is applied, 
\todo{but the smoothing kernel width used is kept at YYYY}
so that the fit to the data can be considered accurate.
\fff energy of 1 is hence
equivalent to that of the QS, and values stronger/weaker 
indicate enhancement/quenching of the \fff w.r.t. QS.
Another difference is that we use the
full ring diagram when computing the \fff energy, while the
other papers used only $k_x=0$ cuts. We have, however, performed
analysis with $k_x=0$ and $k_y=0$ cuts, and see no significant
difference between the two, nor w.r.t. the full ring data, except for the
increased noise level in \ef\,derived from single cuts.
As the measurement is done exactly in the same
manner for the QS and ARs, we anyway anticipate
that this difference should not influence the results significantly. 
Also, as more integration is performed, the less noisy is the 
signal, hence improving the data quality. 

The results from our AR analysis are shown with green symbols and lines in \Fig{AR11130_track} and \Fig{AR11105_track}: in the top panel, each orange point shows the normalized \fff energy, \eft, with four-hour cadence, \todo{smoothed with XXXX over one day to remove a daily fluctuation} otherwise prominently present in the data. The shaded areas show the QS variance, $E'_f$, which is roughly constant, around 4--5 percent, at all latitudes and longitudes. When it is used as normalization of the \fffns, then the uncertainty at large longitudes becomes very high, as is indicated by the rapidly flaring shaded areas toward the limbs. The \brms{} measured simultaneously at the surface is shown in the lowest panels of the figures with black symbols. 

In the middle panels of \Fig{AR11130_track} and \Fig{AR11105_track}, we compare the different calibration curve candidates.
The calibration curve used by 
\cite{SRB16},
based on more simple cosinus dependence of the form $\cos{\alpha} \left[q + \left(1-q \right) \cos{\alpha} \right]$ with $q=0.5$, is shown with blue symbols and lines in the aforementioned panels.
As this specific curve does not fit the QS data very well, we also performed a fit and determined the optimal value of $q$, that turns out to be -0.045; this curve and calibration with it are plotted with orange symbols and lines in the top and middle panel of the figures.
The differences between these calibration curves are illustrated in the middle panels: The \cite{SRB16} calibration
curve does not fit the data, overplotted with gray dots, well at larger longitudes. There, it 
is decreasing less strongly than the data. Near the disk center the amplitude is too low to present the data well.
The new cosinus and Gaussian fits to the QS
data represent it better, with the consequence that the calibration curves decrease more steeply toward the limbs.
The fitted cosinus curve has somewhat larger amplitude than the Gaussian fit. This has the effect that the \eft\,calibrated
with the cosinus fit tends to be always below the QS level. The Gaussian fit calibration results at \eft\,values maximally at the QS level before the AR emergence.

As can be seen
from \Fig{AR11130_track} and \Fig{AR11105_track} top panels, 
applying the new QS caliration curve flattens the signal profile in comparison to the earlier calibration used. For both of the selected ARs, the \fff energy close to the limb is below the QS level to start off with, which could indicate that both of these active regions occur in an environment already non-locally affected by a magnetic field. Applying the earlier calibration curve, a clear enhancement above the QS level is visible for AR11130, while for AR11105 the profile is similar, but the enhancement amplitude remains within the level of QS variability (shaded area). The new calibration shows a more modest enhancement, reaching its peak value at similar times as with the earlier calibration for AR11130, while somewhat earlier for AR11105, but its amplitude is not exceeding QS level significantly. Unfortunately, the possible enhancement of \fff energy of the selected ARs occurs so close to the limb that it is not completely certain whether it is an artefact of the rather uncertain limb calibration data, or a true effect. Obviously, a larger sample size should be studied, including also ARs that emerge closer to the west limb to completely prove/refute this scenario.  


%MJK Summarizing
In summary, our analysis shows that the transient
and weak enhancement signal is very sensitive to the calibration
method used. The new method for the QS calibration developed here tends to show a more flat signal with mild enhancement only in comparison to the earlier calibration method of \cite{SRB16}. Although in general a signature of an enhancement is seen, it remains unclear whether it is an effect caused by the uncertainties close to the limb or a real effect. In the light of the QS data, the magnitude of the enhancement is weaker than the QS variability level, and hence its detection and usage as an AR emergence predictor is not possible with the QS calibration method.
%MJK

%AL added quiet sun f-mode variation plot:
%AL The shaded area in the AR plots at disk center is: 87/2650
%AL: command to create this plot:
%fmodetools.fmode_vs_brms(delta=5,clat=0,notitle=True,show_sigma=False)
\colfig{fmode_vs_brms}{\fff 
%strength 
energy
as a function of \brms{} for the most quiet patches at disk center. The colors indicate the grouping according to the years. The standard deviation for the \fff variation is 87.}


%AL adding two AR plots as placeholders. Needs to be discussed what should be shown
%AL I changed the plots to the Nishant 1=0.5 cosine calibration. 
%AL The shaded areas are different for the two calibrations. 
%AL I cannot believe that this is what Nishant got. I searched for the
%AL error in the cosine calibration but did not find any. What shall we do?
%MJK Is this comment still valid?
%MJK: command to create this plot:
% fmodetools.plot_AR(arcode='11130',quadrant_index=[0,0,0],fit=['COS','COS','GAUSSIAN'],order=[0,1,1],magmode='fits',thumbnails=None,calib=True,tbin=365.25*2,yearly_remove=True,lonrg=[-65,65],plot_label=[r"$\cos$-curve ($q=0.5$)",r"$\cos$-fit ($q=-0.045$)","Gaussian fit"])
\colfig{AR11130_track}{Calibrated \fff strength (\eft, top), QS calibration data (\ef, gray dots) and fits (middle) and \brms\ (bottom) for AR11130. The line-plot colors in the top and the middle panel indicate \eft and \ef for various calibration methods as indicated in the legend. The shaded area in the top panel represents the standard deviation of the QS calibration derived from \fig{fmode_vs_brms} and scaled with the fitted calibration functions.}
\colfig{AR11105_track}{Same as \fig{AR11130_track}, but for AR11105.}
%AL

\section{Discussion}

%NS: added for anticorrelation; could be speculative, so please check and change as needed
%NS: a bit rushed writing as well, but it can be polished later
Anticorrelation of the \fff strength with the solar cycle as seen in \fig{butterfly_lat}
is expected, as the magnetic fields of ARs tend to absorb the modes resulting in the
\fff damping which has been reported in a number of earlier works
\citep{Cally+94,CB97,SRB16}. Transient effects, such as the strengthening of the mode
prior to emergence of ARs as found in \citet{SRB16}, are washed out in such a study aimed
at global behavior of \fff over the whole cycle. Relevant effect here is the damping
caused by the presence of ARs. This is a nonlocal effect where quiet patches in the
vicinity of ARs too display the damping of the \fff. This hypothesis was tested
by a detailed analysis of a magnetically quiet patch next to AR~12529; figures 11 and 12
in \citet{SRB16}. This is further established in the present work, where, though only the
quiet patches are selected in the analysis, damping around the time of solar maximum is
nevertheless observed, as there are a large number of ARs present on the surface during this
phase, thus nonlocally affecting the strengths of the \fff.
%NS.

%NS: adding for now as a subsection, but can be changed later
%NS: hasty writing. will revise later, but essential point i thought is made
%NS: see if you agree, feel free to remove if you don't
%MJK I do agree with the importance of the toroidal subsurface 
%MJK (horizontal) field. In the patches that are used to compute
%MJK the \fff from, there are no AR vertical fields, but 
%MJK magnetic fluctuations from SSD and tangling from the large-scale
%MJK horizontal field. If the latter is stronger, then one might
%MJK expect what you describe as the delayed damping due to the
%MJK late-rejuvenated large-scale field due to the helicity 
%MJK conservation. This would lead to the interesting conclusion
%MJK that perhaps the subsurface toroidal field is somehow much
%MJK stronger during Cycle 25 than Cycle 24. But, instrumental
%MJK degradation could easily produce something similar, hence
%MJK I think we cannot make too strong statements.
%AL The new KXY0 butterflies show exactly this: The toroidal field seems to be
%AL stronger in cycle 25 (reduced \fff around 2021 compared to 2010).

\subsection{Dynamo origin of the shift in the \fff butterfly diagram}
We speculate here on the origin of shift observed in the \fff butterfly diagram
with respect to the solar cycle; see \fig{butterfly_lat}.
Series of earlier numerical and observational works have established that the \fff can be
significantly perturbed in the presence of magnetic fields \citep{S+14,S+15,SRB16,S+20}.
%NS: appears too much of self-citation above; will select only most relevant ones later
In turbulent dynamo theory, the magnetic fields grow under the constraint of magnetic helicity
conservation, which leads to a bihelical spectrum of magnetic helicity. Accumulation of small-scale
magnetic helicity leads to the quenching of large-scale dynamo. In order to further grow its large-scale
magnetic field, the system must shed its small-scale helicity. Being an open system, the Sun
may have fluxes of magnetic helicity where ARs could play a vital role in removing the magnetic
helicity from small-scales, thus leading to a rejuvenation of large-scale dynamo.

It is thus expected in this scenario that the diffuse large-scale component of the magnetic field
becomes stronger \emph{after} the maximum of the magnetic cycle, as the cycle-maximum corresponds
only to the largest number of ARs during that cycle, and not necessarily to the strongest field
strengths. Such diffuse fields will suppress the vertical motions as has been seen in \citet{S+15}
based on idealized simulations, and will therefore further damp the \fff.
To argue more simply, let us consider ARs as sites of vertical fields which maximize during the
solar maximum. On the other hand, diffused fields, that are rejuvenated after the cycle-maximum due
to AR-driven shedding of small-scale magnetic helicity, may be modelled in terms of horizontal
magnetic fields. Whereas both vertical as well horizontal fields lead to the damping of the
\fff, the mechanisms involved are different: AR or the vertical fields couple different layers
of the Sun and act as absorbers of the surface mode power, the horizontal fields simply suppress the
vertical motions leading to lesser amplitudes for the \fff.
%NS.

\begin{acknowledgements}
All SDO data used are publicly available from the Joint Science Operations Center (JSOC) at Stanford University supported by NASA Contract NAS5- 02139 (HMI), see http://jsoc.stanford.edu/. 
The data analysis has been carried out on supercomputers in the facilities hosted by the CSC---IT
Center for Science in Espoo, Finland, which are financed by the
Finnish ministry of education, and in BLAA BLAA in Germany.
\todo{German DataCenter?}
This project has received funding from the European Research Council (ERC)
under the European Union's Horizon 2020 research and innovation
program (Project UniSDyn, grant agreement n:o 818665).

\end{acknowledgements}

\bibliography{main}{}
\bibliographystyle{aasjournal}

\end{document}

%AL BEGIN unnecessary stuff
\color{cyan}
\subsection*{All the stuff below this line not needed !!!}
Error definitions:\\
$\lambda$ = latitude, $t$=time, \\
$N_t$ = number of bins (time), \\
$N_\lambda$ = number of bins (latitude), \\
$N_P$ = number of \fff measurements in a patch, \\
$\overline{X}$ indicates average over time, \\
$f(\lambda,t)$ = \fff strength (unbinned)\\
$f_P(\lambda,t)$ = \fff strength in latitude / time bins (patches)\\
$\overline{f}(\lambda) = 1/N_t \sum_t f(\lambda,t)$ = temporal mean of the \fff strength (unbinned)\\ 
$f'=f-\overline{f}$ is the fluctuation around the temporal mean, i.e. the solar cycle dependence of the \fff (unbinned)\\
$\overline{f_P}(\lambda) = 1/N_t \sum_t f_P(\lambda,t)$ = temporal mean of 
\fff strength (in lat / time bins), \\
$f'_P=f_P-\overline{f_P}$ is, correspondingly, the binned fluctuation\\

%AL added this
The following 4 definitions represent what is plotted in \fig{butterfly_lat_old}:
\begin{itemize}
\item sdev(cycle) 
%AL added the subsript cyc to indicate that this sigma is computed  over the whole cycle, and to distinguish it from \sigma_P
%$\sigma(\lambda)$ 
$\sigma_{\mbox{cyc}}(\lambda)$ 
(red shaded area): this quantity displays the variation of the \fff over the solar cycle.
%AL added
the subscript 'cyc' should indicate that this variation includes also the solar cycle variation. This distinguishes it from $\sigma_P$ defined below, which does not contain the solar cycle variation.
\begin{equation}
%MJK I think subscript P is missing from here, and the need for 'cyc' I cannot understand.
%AL correct. The P was missing. I tried to explain the 'cyc'
%\sigma^2_{\mbox{cyc}}(\lambda) 
%AL commented out this:
%\sigma_P^2(\lambda)  = 
%AL and changed it to:
\sigma^2_{\mbox{cyc}}(\lambda) =
\frac{1}{N_t} \sum_{t} \left( f_P(\lambda,t) -\overline{f_P}(\lambda)\right)^2,
\end{equation}
%MJK added this
hence equalling to the rms value of $f'_P$ (without taking the square root). 
%MJK

\item min/max $\sigma_{\mbox{mm}}$: (blue shaded area): this shows the minimum and the maximum  of $f(\lambda,t) - \overline{f}(\lambda)$:
\begin{equation}
\sigma_{\mbox{mm}}(\lambda) = [ \min_t (f(\lambda,t)),  \max_t (f(\lambda,t)) ] -\overline{f}(\lambda)
\end{equation}

\item min/max (mean) $\overline{\sigma}_{\mbox{mm}}$: (green shaded area): this shows the mean minimum and maximum  of $f(\lambda,t) - \overline{f}(\lambda)$:
\begin{equation}
\overline{\sigma}_{\mbox{mm}}(\lambda) = [ \overline{ \min_t (f(\lambda,t)) }, \overline{ \max_t (f(\lambda,t)) }] -\overline{f}(\lambda)
\end{equation}

\item mean(sdev(patch)) $\overline{\sigma_P}$ (error bars): the standard deviation of the \fff strength
%AL added this
($f(\lambda,t)$)
%AL
 is computed individually for all lat / time bins. This quantity  ($=\sigma_P(\lambda,t)$) is then averaged over time:
\begin{eqnarray}
\overline{\sigma_P}(\lambda) &=&  \frac{1}{N_t} \sum_{t}  \sigma_P(\lambda,t) \mbox{, with} \\
\sigma_P^2(\lambda,t) &=& \frac{1}{N_P}\sum_{\mbox{patch}} ( f(\lambda,t)
- f_P(\lambda,t) )^2,
\end{eqnarray} 
%AL added this
where $\sum_{\mbox{patch}}$ indicates the summation over all \fff strengths within a patch, and $f_P(\lambda,t)$ being the mean \fff strength in this patch. The patch size must be chosen small enough to not contain a significant variation in latitude or solar cycle. 
This quantity $\overline{\sigma_P}(\lambda)$ is displayed as the orange shaded area in \fig{butterfly_lat}.\\
%MJK Added this
This quantity is, therefore, a mean of the $f'_{P, {\rm rms}}$ (so this time also taking the square root) values in individual bins. The problem is that the quantity we are interested in inspecting, and defining the significance of, is now defining the error bar of the measurement. 
%MJK
%AL comment:
%My definition is maybe poor. Yes, I think it should be simply called the 'average rms fluctuation of the \fff strength over 2 months' (in the above plot the patch size in time is 2 months).
\end{itemize}
\subsection*{All the stuff above this line not needed !!!}
\color{black}
%AL END unnecessary stuff

%MJK Cutting this out following the decision from yesterday
%
%AL create this plot with:
% %run plot_fmode_butterfly.py -i ../qs_cubes --reread=False --fov 'FFOV_KX0' --show_cos=False --show_cycle=False
\colfigtwocol{butterfly_lat_FFOV_KX0}{Same as \fig{butterfly_lat}, but computed from a wedge of $\pm10^\circ$ around the $k_y$ axis.
%MJK I would not know what is a poloidal component of the f-mode.
%, representing the poloidal component.
\label{butkx0}}

\colfigtwocol{butterfly_lat_FFOV_KY0}{Same as \fig{butterfly_lat}, but computed from a wedge of $\pm10^\circ$ around the $k_x$ axis.
%, representing the toroidal component. 
\label{butky0}}

%MJK
Next we investigate whether constructing the \fff 
%power 
energy
near
the ring axes will affect the results. Such a difference could be caused by the underlying magnetic field running in a preferred direction. 
%MJK
\Fig{butkx0} shows the butterfly near the $k_x$=0 axis, $f_{k_{y}}$, tracing \fff sensitivity on the poloidal component of the
magnetic field, and \Fig{butky0} the same near the $k_y$=0 axis,
$f_{k_{x}}$, tracing the sensitivity to the toroidal field. Interestingly,
differences can be observed. First of all, the variability in
$f_{k_{x}}$ is somewhat stronger then in $f_{k_{y}}$, that
could indicate that the subsurface toroidal field is stronger
than the preferentially poloidal one. Secondly, $f_{k_{x}}$ is
better in phase with the solar cycle, although some phase shift,
but smaller, is observed. The signal is the most pronounced
at the equator. It does not increase during the 24-to-25
minimum except at higher latitudes, and no signs of quenching
are yet clearly discernible during the ascending phase of Cycle 25. 

$f_{k_{x}}$ has a clear and larger phase shift w.r.t. the solar 
cycle, especially pronounced at high latitudes. In the beginning
of the HMI data series, clear signs of quenching are still
visible at high latitudes on both hemispheres. Then enhancement is 
seen around the the solar maximum. This is followed by weak signs
of quenching during the descending and minimum phase. Enhancement
is again observed during the ascending phase of Cycle 25, especially
in the North. The Southern hemisphere shows generally weaker
patterns than the North in all 
computed
\fff 
%components.
variants.