
Dear language editor,

thanks for careful reading of our manuscript. We wonder how
quiet Sun XX (XX being for example magnetism) should finally be spelled out, as now the title and running tile have it with hyphen, while the main text without. Could you still please check that? Concerned with this, we have corrected two occurrences of quiet-Sun XX from Figure 3 caption.

We also note that the reference list included is not the one of our manuscript. We think that this error in not ours (we have included the correct bib.file), and hope that it will appear correct in the final manuscript.

Our more detailed responses:

Abstract: Abbreviation QS is used, but it is never introduced in the abstract, although it is used there. This could be done on the first line:

The origins of quiet Sun magnetism -> The origins of quiet Sun (QS) magnetism 

Abstract: please change

The solar cycle dependence on the supergranular scale...
->
The dependence of $E_f$ on the solar cycles at supergranular scale...

2.2 Quiet region selection:

Please revert back to past tense, as this was one very early data
analysis and selection steps, leading to the construction of the final
pipeline and data products.

Also, it depicts the low-field internetwork regions very well.

->

Also, it depicted the low-field internetwork regions very well.

When reformulating the bullet points in the same section, we noticed use of confusing language from our side. We would like to change the reason for the usage of criterion (i) to

...to get the strongest magnetic field signal along the central meridian; 

->

...because the central meridian offers the highest sensitivity for magnetic field measurements,

2.5 Computation of the f-mode energy:

Please revert back, as the meaning is changed:

leading to a smoothing of the one-dimensional (1D) $k-\omega$ spectra

->

leading to smooth one-dimensional (1D) $k-\omega$ spectra

Results, first paragraph:

Here we want to refer to the area, not to the previous sentence.
Hence, we propose to change this back.
...this corresponds to an area of $\approx 180 \times 180$\,Mm$^2$, 
which 

->

and 

therefore contains 30--40 supergranular cells.

3.2. AR f-mode with QS calibration, 3rd para

We feel "strongly" would still describe our observation the best.

There, we can see it is decreasing less 

significantly

->
strongly 

than shown by the data.

%AL begin
In the first paragraph of subsection 3.2 the LE suggests to use the present perfect tense:

"Another difference is that we use the
full ring diagram when computing the \fff energy, while the
other papers used only $k_x=0$ cuts. We 
\textsl{have, however, performed}
analyses with $k_x=0$ and $k_y=0$ cuts and 
\textsl{have seen} 
no significant
difference between the two, nor with regard to the full ring data, except for the
increased noise level in \ef\,derived from single cuts."

We think it would be more cnsistent to use the past tense here, and replace the predicate accordingly to:

"Another difference is that we use the
full ring diagram when computing the \fff energy, while the
other papers used only $k_x=0$ cuts. We 
\textsl{However, we performed}
analyses with $k_x=0$ and $k_y=0$ cuts and 
\textsl{saw} 
no significant
difference between the two, nor with regard to the full ring data, except for the
increased noise level in \ef\,derived from single cuts."
%AL end

%AL begin
4.1 Discussion and conclusion

In the second paragraph the LE writes:

" ... or then the sensitivity of the HMI instrument prevents  any variations from BEINGS seen."

We think that "BEING" is correct and replaced "BEINGS".

Fig 6, 11:

The LE complained about the usage of the "Same as..." in the caption. We agree that having an introductory sentence is a better style than starting the caption with "Same as", but the duplication of the label description is confusing. It leaves it up to the reader to "hunt" for discrepancies between the two descriptions, instead of making it clear by stating something like: "The layout of the plot is the same as for Fig. XX", or "Lines, colors, and symbols are the same as in Fig. XX."

%AL end
