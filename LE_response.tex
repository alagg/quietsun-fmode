\documentclass[a4]{article}
\usepackage{xcolor}
\usepackage{graphicx}

\definecolor{cmt}{rgb}{0.5,0.0,0.0}
\definecolor{al}{rgb}{0.6,0.2,0.0}
\definecolor{mk}{rgb}{0.4,0.4,0.0}
\definecolor{no}{rgb}{0,0.0,0.4}
\definecolor{ns}{rgb}{0.4,0.0,0.0}
\definecolor{hr}{rgb}{0.4,0.0,0.0}

\newcommand{\al}[1]{{\color{al}AL: #1}}
\newcommand{\mk}[1]{{\color{mvn}MJK: #1}}
\newcommand{\no}[1]{{\color{corr}NO: #1}}
\newcommand{\ns}[1]{\textsc{\color{al}NS: #1}} 
\newcommand{\hr}[1]{\textbf{\color{ref}HR: #1}} 
\newcommand{\todo}[1]{{\color{cmt}ToDo: #1}}
\newcommand{\cmt}[1]{
{\color{cmt}#1}
}

%\renewcommand{\fig}[1]{Fig.~\ref{#1}} 
\newcommand{\fig}[1]{Fig.~\ref{#1}} 
\newcommand{\Fig}[1]{Figure~\ref{#1}} 
\newcommand{\eq}[1]{Eq.~\ref{#1}} 
\newcommand{\figs}[1]{Figs.~\ref{#1}} 
\newcommand{\tab}[1]{Tab.~\ref{#1}} 
\newcommand{\Tab}[1]{Table~\ref{#1}} 
\newcommand{\sect}[1]{Sec.~\ref{#1}} 
\newcommand{\Sect}[1]{Section~\ref{#1}} 
\newcommand{\secs}[1]{Sections~\ref{#1}} 
\newcommand{\app}[1]{App.~\ref{#1}} 


\newcommand{\kms}{km\,s$^{-1}$}
\newcommand{\ms}{m\,s$^{-1}$}
\newcommand{\grad}{$^\circ$}
\newcommand{\carcsec}{$\mbox{.\hspace{-0.5ex}}^{\prime\prime}$}
\newcommand{\halpha}{H$\alpha$}
\newcommand{\hminus}{H$^-$}
\newcommand{\hei}{He\,\textsc{i}}
\newcommand{\sii}{Si\,\textsc{i}}
\newcommand{\fei}{Fe\,\textsc{i}}
\newcommand{\tii}{Ti\,\textsc{i}}
\newcommand{\cai}{Ca\,\textsc{i}}
\newcommand{\caii}{Ca\,\textsc{ii}}
\newcommand{\caiih}{Ca\,\textsc{ii\,h}}
\newcommand{\coi}{Co\,\textsc{i}}
\newcommand{\helixp}{\textsc{HeLIx$^+$}}
\newcommand{\imax}{{IMaX}}
\newcommand{\sufi}{\textsc{SuFI}}
\newcommand{\sunrise}{\textsc{Sunrise}}
\newcommand{\los}{${\rm LOS}$}
\newcommand{\vlos}{$v_{\rm LOS}$}

\newcommand{\IN}{internetwork}
\newcommand{\inw}{\textsl{INw}}
\newcommand{\nw}{\textsl{Nw}}
\newcommand{\brms}{$B_{\rm RMS}$}

\newcommand{\colfig}[3][1.]{\begin{figure}\centering
    \includegraphics[width=#1\linewidth,clip=TRUE]{#2}
    \caption{#3}
    \label{#2}
\end{figure}}
\newcommand{\colfigtwocol}[3][1.]{\begin{figure*}\centering
    \includegraphics[width=#1\linewidth,clip=TRUE]{#2}
    \caption{#3}
    \label{#2}
\end{figure*}}


\begin{document}

\noindent Dear language editor,\\

\noindent thanks for the careful reading of our manuscript. We do have some major and minor comments, suggestions and edits as a response, that we detail below.\\

\subsubsection*{Our major points:}

\noindent Firstly, we wonder how "quiet Sun XX" (XX being for example magnetism) should finally be spelled out, as now the title and running tile have it with hyphen, while the main text without. We used the hyphen when QS is used as an adjective (e.g., "quiet-sun magnetism") and without a hyphen when it is used as a separate entity (e.g., "the magnetism of the quiet Sun").
Related with this, we have corrected two occurrences of quiet-Sun XX to use abbreviation QS XX in Figure 3 caption. 
We also changed the title of section 3.1 to "Quiet-sun f-mode butterfly diagram" (hyphen added).
We would appreciate if you could check the correctness and the consistency of the usage of the hyphen.\\

\noindent We also note that the reference list included is not the one of our manuscript. We think that this error is not ours (we have included the correct bib.file), and hope that it will appear instead in the final manuscript.\\

\noindent Captions of Figs. 6 and 11:

\noindent You complained about the usage of the "Same as..." in the caption. We agree that having an introductory sentence is a better style than starting the caption with "Same as", but the duplication of the label description is confusing. It leaves it up to the reader to "hunt" for discrepancies between the two descriptions, instead of making it clear by just stating the differences. We changed the captions accordingly: \\
Fig. 6: "\brms{} determined from the most quiet patches from 2010 to 
2022 at disk center, but in contrast to Fig.~5, the patch size was 1$^\circ$ in longitude and latitude, corresponding to \IN{} patches. The color and labeling scheme is the same as in Fig.~5. For all values of $\omega,$ the magnitude of $P$ is always an order of magnitude higher than the noise level outside the considered location of the ring."\\
Fig. 11: "Calibrated \fff strength (\eft, top), QS calibration data (\ef, gray dots) and fits (middle), and \brms\ (bottom) for AR11105. The color and labeling scheme is the same as in Fig.~10."

\subsubsection*{Our more detailed responses:}

\noindent Abstract: Abbreviation QS is used, but it is never introduced in the abstract, although it is used there. This could be done on the first line:

\begin{verbatim}
The origins of quiet Sun magnetism 
-> 
The origins of quiet Sun (QS) magnetism \end{verbatim}

\noindent Abstract: the following sentence has altered its meaning due to the language editing. Please change

\begin{verbatim}
The solar cycle dependence on the supergranular scale...
->
The dependence of $E_f$ on the solar cycles at supergranular scale...
\end{verbatim}

\noindent 2.2 Quiet region selection: \\

\noindent Please revert back to past tense, as this was one very early data
analysis and selection steps, leading to the construction of the final
pipeline and data products.

\begin{verbatim}
Also, it depicts the low-field internetwork regions very well.
->
Also, it depicted the low-field internetwork regions very well.
\end{verbatim}

\noindent When reformulating the bullet points in the same section, we noticed use of confusing language from our side. We would like to change the reason for the usage of criterion (i) to

\begin{verbatim}
...to get the strongest magnetic field signal along the central meridian;
->
...because the central meridian offers the highest sensitivity for 
magnetic field measurements,
\end{verbatim}

\noindent 2.5 Computation of the f-mode energy:

\noindent Please revert back, as the meaning is changed:

\begin{verbatim}
leading to a smoothing of the one-dimensional (1D) $k-\omega$ spectra
->
leading to smooth one-dimensional (1D) $k-\omega$ spectra
\end{verbatim}

\noindent Results, first paragraph:

\noindent Here we want to refer to the area, not to the previous sentence. Hence, we propose to change this back.
\begin{verbatim}
...this corresponds to an area of $\approx 180 \times 180$\,Mm$^2$, 

which 
->
and 

therefore contains 30--40 supergranular cells.
\end{verbatim}

\noindent 3.2. AR f-mode with QS calibration, 3rd para

\noindent We feel "strongly" would still describe our observation the best.

\begin{verbatim}
There, we can see it is decreasing less 

significantly

->
strongly 

than shown by the data.
\end{verbatim}

\noindent Subsection 3.2, first paragraph:\\
You suggested to use the present perfect tense:

"Another difference is that we use the
full ring diagram when computing the \fff energy, while the
other papers used only $k_x=0$ cuts. We 
\textbf{have, however, performed}
analyses with $k_x=0$ and $k_y=0$ cuts and 
\textbf{have seen} 
no significant
difference between the two, nor with regard to the full ring data, except for the
increased noise level in \ef\,derived from single cuts."

\noindent We think it would be more consistent to use the past tense here. We therefore replaced the predicate accordingly to:

"Another difference is that we use the
full ring diagram when computing the \fff energy, while the
other papers used only $k_x=0$ cuts. We 
\textbf{However, we performed}
analyses with $k_x=0$ and $k_y=0$ cuts and 
\textbf{saw} 
no significant
difference between the two, nor with regard to the full ring data, except for the
increased noise level in \ef\,derived from single cuts."\\

\noindent 4.1 Discussion and conclusions\\
\noindent In the second paragraph you corrected one of our sentences to:\\
\noindent " ... or then the sensitivity of the HMI instrument prevents  any variations from BEINGS seen."\\
\noindent We think that "BEING" is correct and replaced "BEINGS".

\end{document}